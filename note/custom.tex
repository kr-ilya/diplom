% Здесь указываются пользовательские стили, команды...
% Также возможно переопределение стилей, команд

% Стилизация кода

\setmonofont{DejaVu Sans Mono}

\lstset{basicstyle=\ttfamily\large,
	breaklines=true,
	keepspaces=true,
	frame=single,
	xleftmargin=.25\textwidth,
	xrightmargin=.25\textwidth,
	showstringspaces=false
}

\usepackage{array}

% Установить рамки.
% По умолчанию рамки и так установлены. Если вы хотите их убрать,
% то установите false. Можно удалить или закомментировать данную 
% строчку - рамки останутся.
% Данная опция полезна, когда нужно конвертировать pdf в docx для отправки на
% антиплагиат, чтобы избавиться от таблиц, в которые превращаются рамки.
\setboolean{setframes}{true} % or false

% Отключение переносов слов
% \usepackage[none]{hyphenat}
% \hyphenpenalty=10000
% \hbadness=10000
\newpage

\csection{Заключение}

В ходе выполнения выпускной квалификационной работы была проведена разработка предметно-ориентированного языка для конструктора Telegram ботов и интерпретатора для него.
Основной целью проекта было расширение функциональных возможностей визуального конструктора Telegram ботов.
Благодаря использованию DSL можно более гибко настроить логику работы бота и предоставить пользователю платформы возможность разработки сложных, уникальных ботов,
которых было бы невозможно или затруднительно построить с использованием только базового набора компонентов визуального конструктора.

На основе анализа предметной области и определения основных функциональных возможностей предметно-ориентированного языка было сформировано расширенное технического задание,
определяющее требования к разрабатываемому продукту.

Исходя из требований технического задания первоочередной задачей было разработать и описать грамматику предметно-ориентированного языка.
Для описания формальной грамматики была выбрана расширенная форма Бэкуса-Наура.

Была выполнена разработка архитектурно-структурных решений и алгоритмов функционирования каждой составляющей модуля интерпретации,
а именно лексического, синтаксического, семантического анализаторов и исполнителя инструкций предметно-ориентированного языка.
Также описан программный интерфейс интеграции модуля с серверной частью конструктора.

На этапе программной реализации были выполнено кодирование анализаторов и исполнителя с помощью выбранных инструментов разработки
в соответствии со структурными решениями и алгоритмами функционирования.

Кроме этого было осуществлено модульное тестирование разработанной программы, в ходе которого все тесты были выполнены успешно.
Также выполнена экспериментальная апробация с примером использования разработанного программного продукта.

Разработанный предметно-ориентированный язык и модуль интерпретации для него решают проблему ограниченных функциональных возможностей конструктора Telegram ботов.
Теперь пользователь визуального конструктора может реализовывать множество уникальных идей из различных предметных областей,
не ограничиваясь стандартным набором компонентов.

Исходный код разработанного модуля интерпретации выложен в публичный доступ на Github под открытой лицензией MIT.
Открытый исходный код под данной лицензией предоставляет возможность любому пользователю вносить изменения, разворачивать решение на своей стороне и эксплуатировать в коммерческих целях.

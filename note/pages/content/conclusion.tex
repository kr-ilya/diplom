\newpage

\csection{Заключение}

В ходе выполнения выпускной квалификационной работы выполнена разработка предметно-ориентированного языка для конструктора Telegram ботов и интерпретатора для него.
Основной целью проекта было расширение функциональных возможностей визуального конструктора Telegram ботов.
Благодаря использованию DSL можно более гибко настроить логику работы бота и предоставить пользователю платформы возможность построения сложных, уникальных ботов,
которых было бы невозможно или затруднительно построить с использованием только базового набора компонентов визуального конструктора.

На основе анализа предметной области и определения основных функциональных возможностей предметно-ориентированного языка было сформировано расширенное технического задание,
определяющее требования к разрабатываемому продукту.

Исходя из требований технического задания первоочередной задачей было разработать и описать грамматику предметно-ориентированного языка.
Для описания формальной грамматики была выбрана расширенная форма Бэкуса-Наура.

Была выполнена разработка архитектурно-структурных решений и алгоритмов функционирования каждой составляющей части модуля интерпретации,
а именно лексического, синтаксического, семантического анализаторов и исполнителя инструкций предметно-ориентированного языка.
Также описан программный интерфейс интеграции модуля с серверной частью конструктора.

На этапе программной реализации были выполнено кодирование анализаторов и исполнителя
в соответствии с разработанными алгоритмами функционирования с помощью выбранных инструментов разработки.
Кроме этого было осуществлено модульное тестирование разработанной программы.

Разработанный предметно-ориентированный язык позволяет расширить функциональные возможности конструктора Telegram ботов.
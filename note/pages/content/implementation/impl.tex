\newpage

\section{Программная реализация}

В данном разделе представлено описание выбранных инструментов разработки для решения поставленной задачи. 
Описана программная реализация модуля интерпретации предметно-ориентированного языка.
Рассмотрены детали реализации его основных структурных частей.
Также выполнено тестирование основных функций программы.

\subsection{Выбор инструментов разработки}

В качестве языка программирования для разработки был выбран Go.

Go (Golang) -- это компилируемый многопоточный язык программирования с открытым исходным кодом, разработанный компанией Google \refref{ref:golang}.

Выбор данного языка обусловлен совместимостью между разрабатываемым модулем и основной системой,
так как для реализации серверной части конструктора Telegram ботов так же был выбран язык Go.
Использование одного языка для всей системы гарантирует,
что модули будут легко интегрироваться друг с другом без необходимости создания сложных интерфейсов или адаптеров.

Язык Go имеет и другие положительные характеристики:
\begin{itemize}
    \item высокая производительность -- Go компилируется в машинный код, что обеспечивает его высокую скорость исполнения;
    \item безопасность типов -- строгая статическая типизация предотвращает множество ошибок на этапе компиляции;
    \item встроенная поддержка параллелизма -- в Go реализованы горутины и каналы -- встроенные механизмы для эффективной работы с параллельными задачами;
    \item расширяемость -- язык имеет встроенную систему управления модулями и широкую экосистему общедоступных библиотек.
\end{itemize}

Выбор языка программирования Go для реализации модуля интерпретации предметно-ориентированного языка конструктора Telegram ботов обоснован тем,
что он позволяет поддерживать единообразие кода, использовать его технические преимущества и соответствовать основным требованиям проекта.

\subsubsection{Реализация лексического анализатора}

Лексический анализатор состоит из следующих основных компонентов:
\begin{itemize}
    \item токены -- определение структуры и типов токенов, представляющих основные элементы языка;
    \item конечный автомат -- принимает на вход поток символов и определяет токены переходя между состояниями в соответствии с правилами языка.
\end{itemize}

Токен представляет собой структуру, содержащую информацию о типе токена и его значение, представленное в виде строки.
Код структуры, представляющей токен приведен на рисунке~\ref{f:code_tokenStruct}.

\begin{figure}[ht]
	\centering
	\vspace{\toppaddingoffigure}
	\begin{lstlisting}[
        language=Go
    ]
type TokenType = string
type Token struct {
    Type    TokenType
    Literal string
}        
\end{lstlisting}
	\caption{Структура, представляющая токен}
	\label{f:code_tokenStruct}
\end{figure}

Список возможных токенов представлен в таблице~\ref{t:tokens}.

Программную реализацию токенов можно выполнить с помощью списка константных значений.
Фрагмент кода реализации токенов представлен на рисунке~\ref{f:code_tokensFragemnt}.

\begin{figure}[ht]
	\centering
	\vspace{\toppaddingoffigure}
	\begin{lstlisting}
IDENT  = "IDENT"  // x, t, add
INT    = "INT"    // 123
STRING = "STRING" // "abcde"
ASSIGN = "="
PLUS   = "+"
STAR   = "*"

// keywords
IF     = "IF"
ELSE   = "ELSE"
TRUE   = "TRUE"    
\end{lstlisting}
	\caption{Фрагмент кода реализации токенов}
	\label{f:code_tokensFragemnt}
\end{figure}

Лексер представляет собой структуру, содержащую информацию о входной строке кода,
текущем считанном символе, позиции курсора и других технических значениях, необходимых для корректной работы анализатора.
Структура представляющая лексер приведена на рисунке~\ref{f:code_lexerStruct}.

\begin{figure}[ht]
	\centering
	\vspace{\toppaddingoffigure}
	\begin{lstlisting}[
        language=Go,
        xleftmargin=.08\textwidth,
        xrightmargin=.08\textwidth
    ]
type Lexer struct {
    input   string
    ch      byte // current char
    pos     int  // current position (on current char)
    readPos int  // position after current char
    nlsemi  bool // if "true" '\n' translate to ';'
    loPos   token.Pos
}    
\end{lstlisting}
	\caption{Структура, представляющая токен}
	\label{f:code_lexerStruct}
\end{figure}

Основная функция лексического анализатора может быть реализована в формате конечного автомата.
При получении очередного символа из входной строки кода его необходимо сопоставить с одним из токенов.
Стоит заметить, что некоторые токены формируются за счет двух и более символом, например токен <<LAND>> (\&\&), идентификаторы, ключевые слова и т.д. 
В этом случае, необходимо продолжать получение символов из входной строки до тех пор, пока не будет однозначно определен токен.

Основная функция определения токена выполнена в виде конструкции switch-case.
Фрагмент кода представлен на рисунке~\ref{f:code_lexerFragment}.

% Полный код лексера представлен в приложении ...

\clearpage

\begin{figure}[!ht]
	\centering
	\vspace{\toppaddingoffigure}
    \begin{lstlisting}[
        language=Go,
        xleftmargin=.08\textwidth,
        xrightmargin=.08\textwidth
    ]
func (l *Lexer) NextToken() (token.Token, token.Pos) {
    l.skipWhitespace()
    nlsemi := false
    var tok token.Token
    switch l.ch {
    case '\n':
        tok = newToken(token.SEMICOLON, l.ch)
    case '=':
        if l.peekChar() == '=' {
            l.readChar()
            literal := "=="
            tok = token.Token{Type: token.EQ, Literal: literal}
        } else {
            tok = newToken(token.ASSIGN, l.ch)
        }
    case '+':
        tok = newToken(token.PLUS, l.ch)
    case '-':
        tok = newToken(token.MINUS, l.ch)
    case '*':
        tok = newToken(token.STAR, l.ch)
    case '/':
        tok = newToken(token.SLASH, l.ch)
    case '!':
        if l.peekChar() == '=' {
            l.readChar()
            literal := "!="
            tok = token.Token{Type: token.NEQ, Literal: literal}
        } else {
            tok = newToken(token.EXCLAMINATION, l.ch)
        }
    case '%':
		tok = newToken(token.PERCENT, l.ch)
	case '<':
		if l.peekChar() == '=' {
			l.readChar()
			literal := "<="
			tok = token.Token{Type: token.LEQ, Literal: literal}
		} else {
			tok = newToken(token.LT, l.ch)
		}
\end{lstlisting}
	\caption{Фрагмент кода лексера}
	\label{f:code_lexerFragment}
\end{figure}
\subsubsection{Реализация синтаксического анализатора}

Разработка синтаксического анализатора включает в себя программную реализацию парсера, способного анализировать токены, получаемые от лексического анализатора и строить абстрактное синтаксическое дерево.

Абстрактное синтаксическое дерево представляет собой структуру, отражающую синтаксическую структуру программы.
Узлы AST могут быть двух типов: statement -- инструкции и expression -- выражения. 
В соответствии с этим, их программная реализация выполнена в виде интерфейсов, представленных на рисунке~\ref{f:code_astInterfaces}.

\begin{figure}[ht]
	\centering
	\vspace{\toppaddingoffigure}
	\begin{lstlisting}[
        language=Go
    ]
type Node interface {
    TokenLiteral() string
    ToString() string
}

// All statement nodes implement
type Statement interface {
    Node
    statementNode()
}

// All expression nodes implement
type Expression interface {
    Node
    expressionNode()
} 
\end{lstlisting}
	\caption{Интерфейсы узлов AST}
	\label{f:code_astInterfaces}
\end{figure}

Узлы дерева состоят из интерфейса Node.
Однако сам по себе он не используется в AST, а необходим для расширения двух вспомогательных интерфейсов Statement и Expression, которые определяют узлы двух типов: инструкции и выражения соответственно.

Пример кода структуры AssignStatement, реализующей интерфейс Statement на рисунке~\ref{f:code_IStatementExample}.

Пример кода структуры IntegerLineral, реализующей интерфейс Expression представлен на рисунке~\ref{f:code_IExpressionExample}.

\begin{figure}[!htb]
	\centering
	\vspace{\toppaddingoffigure}
	\begin{lstlisting}[
        language=Go,
        xleftmargin=.08\textwidth,
        xrightmargin=.08\textwidth
    ]
type AssignStatement struct {
    Name  *Ident
    Value Expression
}

func (as *AssignStatement) statementNode()       {}
func (as *AssignStatement) TokenLiteral() string { return "" }
func (as *AssignStatement) ToString() string {
    var out bytes.Buffer

    out.WriteString(as.Name.TokenLiteral())
    out.WriteString(" = ")

    if as.Value != nil {
        out.WriteString(as.Value.ToString())
    }

    out.WriteString(";")

    return out.String()
}
\end{lstlisting}
	\caption{Пример реализации интерфейса Statement}
	\label{f:code_IStatementExample}
\end{figure}

\begin{figure}[!htb]
	\centering
	\vspace{\toppaddingoffigure}
	\begin{lstlisting}[
        language=Go,
        xleftmargin=.08\textwidth,
        xrightmargin=.08\textwidth
    ]
type IntegerLiteral struct {
    Token token.Token // 5 6
    Value int64
}

func (il *IntegerLiteral) expressionNode()      {}
func (il *IntegerLiteral) TokenLiteral() string { return il.Token.Literal }
func (il *IntegerLiteral) ToString() string     { return il.Token.Literal }
\end{lstlisting}
	\caption{Пример реализации интерфейса Expression}
	\label{f:code_IExpressionExample}
\end{figure}

Рассмотрим пример работы синтаксического анализатора.

Входная строка: $5 + 1 * 2 / (4 + 9)$.

Результат работы синтаксического анализатора в виде строки с исходным кодом программы,
в котором с помощью скобок обозначены приоритеты операторов представлена на рисунке~\ref{f:parserCodeResult}.

\begin{figure}[ht]
	\centering
	\vspace{\toppaddingoffigure}
	\includegraphics[width=0.7\textwidth]{parser/parserCodeResult.png}
	\caption{Результат работы синтаксического анализатора в виде строки}
	\label{f:parserCodeResult}
\end{figure}

Графическое представление AST для указанных входных данных представлено на рисунке~\ref{f:ast}.
Дерево, сформированное в результате работы программы для указанных входных данных представлен на рисунке~\ref{f:astRawCmd}.

\begin{figure}[ht]
	\centering
	\vspace{\toppaddingoffigure}
	\includegraphics[width=0.7\textwidth]{parser/ast.pdf}
	\caption{AST для указанной входной строки}
	\label{f:ast}
\end{figure}

\clearpage

\begin{figure}[!htb]
	\centering
	\includegraphics[width=0.48\textwidth]{parser/astRawCmd.pdf}
	\caption{Результат работы парсера в виде AST}
	\label{f:astRawCmd}
\end{figure}

\clearpage
\subsubsection{Реализация семантического анализатора}

Прежде, чем переходить непосредственно к реализации семантического анализатора, необходимо реализовать объектную систему.
Объектная система является основополагающей частью семантического анализатора и исполнителя.
Семантической анализатор в процессе своей работы выполняет необходимые проверки на основе значений,
представленных в виде объектов внутреннего представления.

Каждое значение выражения представляется в виде структуры, которая соответствует некоторому объекту интерфейсного типа -- рисунок~\ref{f:code_ObjectInterface}.

\begin{figure}[ht]
	\centering
	\vspace{\toppaddingoffigure}
	\begin{lstlisting}[
        language=Go
    ]
type ObjectType string

type Object interface {
    Type() ObjectType
    ToString() string
}
\end{lstlisting}
	\caption{Интерфейс объекта}
	\label{f:code_ObjectInterface}
\end{figure}

На рисунке~\ref{f:code_IObjectExample} приведен пример структуры, которая представляет данные типа «целое число».

\begin{figure}[ht]
	\centering
	\vspace{\toppaddingoffigure}
	\begin{lstlisting}[
        language=Go,
        xleftmargin=.08\textwidth,
        xrightmargin=.08\textwidth
    ]
func (i *Integer) Type() ObjectType { return INTEGER_OBJ }
func (i *Integer) ToString() string {
    return fmt.Sprintf("%d", i.Value)
}
func (i *Integer) HashKey() HashKey {
    return HashKey{Type: i.Type(), Value: uint64(i.Value)}
}
\end{lstlisting}
	\caption{Пример реализации интерфейса Object}
	\label{f:code_IObjectExample}
\end{figure}

Поле «Value» предназначено для хранения значения числа.
Метод «Type()» возвращает информацию о принадлежности структуры типу «integer». 
Метод «ToString()» формирует хранимое значение в виде строки.
Используется для читаемого представления значения объекта в процессе отладки. 

Подобные структуры, реализующие интерфейс «Object» представлены для всех примитивных и составных типов данных, используемых в языке:
boolean, string, integer, array, HashMap.
Кроме этого, реализованы еще несколько вспомогательных структур:

\begin{itemize}
    \item «Null» для поддержки соответствующих значений;
    \item «Error», содержащая информацию об ошибке, возникшей на этапе семантического анализа;
    \item «Return» для представления возвращаемых значений;
    \item «Function» - специальная структура, используемая при обработке вызова функции;
    \item «Builtin» - структура, представляющая встроенные функции.
\end{itemize}

Данные объекты формируют объектную систему внутреннего представления значений программы.

Семантический анализ выполняется во время рекурсивного прохода по узлам AST.
В ходе рекурсии при достижении примитивных значений в крайних узлах ветвей AST для каждого значения создается объект  внутреннего представления соответствующего типа.
На обратном ходу рекурсии, при необходимости выполнить проверку семантической корректности выражения она выполняется над объектами внутреннего представления, сформированными ранее.
При анализе генерируется ошибка, в случае её обнаружения, в противном случае начинается вычисление значение выражения исполнителем.

Фрагмент кода функции разбора инфиксного выражения представлен на рисунке~\ref{f:code_semantic}.

\begin{figure}[ht]
	\centering
	\vspace{\toppaddingoffigure}
	\begin{lstlisting}[
        language=Go,
        xleftmargin=.08\textwidth,
        xrightmargin=.03\textwidth
    ]
func evalInfixExpression(op string, left object.Object, right object.Object) object.Object {
    switch {
    case left.Type() != right.Type():
        return newError("type mismatch: %s %s %s", left.Type(), op, right.Type())
    case left.Type() == object.INTEGER_OBJ && right.Type() == object.INTEGER_OBJ:
        return evalIntInfixExpr(op, left, right)
    case left.Type() == object.STRING_OBJ && right.Type() == object.STRING_OBJ:
        return evalStringInfixExpr(op, left, right)
    case op == "==":
        return boolToBooleanObj(left == right)
    case op == "!=":
        return boolToBooleanObj(left != right)
    case op == "||":
        return boolToBooleanObj(left.(*object.Boolean).Value || right.(*object.Boolean).Value)
    case op == "&&":
        return boolToBooleanObj(left.(*object.Boolean).Value && right.(*object.Boolean).Value)
    default:
        return newError("unknown operator: %s %s %s", left.Type(), op, right.Type())
    }
}
\end{lstlisting}
	\caption{Пример семантического анализа инфиксного выражения}
	\label{f:code_semantic}
\end{figure}
\subsection{Реализация исполнителя}

В общем виде процесс исполнения тесно связан с этапом семантического анализа.
Выполняется рекурсивный обход абстрактного синтаксического дерева.
Первым шагом каждое выражение проходит семантическую проверку.
После успешного завершения семантического анализа выражения из AST передаются на этап их вычисления.
Разнотипные выражения обрабатываются по-разному, однако результат вычисления всегда представляет собой некоторый тип данных, представленный в виде объекта -- внутреннего представления.

Окружение для хранения информации о переменных в программном коде реализовано в виде структуры,
содержащей поля с хэш-картой с самими переменными и их значениями, а также ссылка на эту же структуру для организации области видимости при вызове функций.
На рисунке~\ref{f:code_envStruct} приведен пример программной реализации данной структуры.

\begin{figure}[ht]
	\centering
	\vspace{\toppaddingoffigure}
	\begin{lstlisting}[
        language=Go
    ]
type Env struct {
    store map[string]Object
    outer *Env
}
\end{lstlisting}
	\caption{Реализация окружения}
	\label{f:code_envStruct}
\end{figure}

За обработку узлов AST отвечает единственная функция, которая определяет тип узла и передает управление соответствующей функции,
которая после прохождения семантической проверки вычисляет значение для узла данного типа.
В конечном итоге формируется внутреннее представление в виде объектов с примитивными типами данных, либо объекты представляющие составные типы, состоящие из примитивных, так как массивы и хеш-карты.
Фрагмент кода основной функции получения и обработки узлов AST приведен на рисунке~\ref{f:code_evalFragment}.

При обнаружении в коде определения переменной, ее необходимо сохранить в памяти, чтобы в дальнейшем иметь к ней доступ.
Для этого используется окружение.
Переменные в окружении хранятся в виде хэш-карты, ключи которой представляют идентификатор переменной, а значения – внутреннее представление значений переменной, то есть объекты.

Код функций записи переменной в окружение и получения из него приведен на рисунке~\ref{f:code_getsetEnv}.

\clearpage

\begin{figure}[!htb]
	\centering
	\begin{lstlisting}[
        language=Go,
        xleftmargin=.08\textwidth,
        xrightmargin=.08\textwidth
    ]
func Eval(n ast.Node, env *object.Env) object.Object {
    switch node := n.(type) {
    case *ast.Program:
        return evalProgram(node, env)
    case *ast.BlockStatement:
        return evalBlockStatement(node, env)
    case *ast.ExpressionStatement:
        return Eval(node.Expression, env)
    case *ast.ReturnStatement:
        val := Eval(node.Value, env)
        if isError(val) {
            return val
        }

        return &object.Return{Value: val}
    case *ast.AssignStatement:
        val := Eval(node.Value, env)
        if isError(val) {
            return val
        }

        env.Set(node.Name.Value, val)
    case *ast.IntegerLiteral:
        return &object.Integer{Value: node.Value}
    case *ast.Boolean:
        if node.Value {
            return TRUE
        }
        return FALSE
\end{lstlisting}
	\caption{Фрагмент кода исполнителя}
	\label{f:code_evalFragment}
\end{figure}

\begin{figure}[!htb]
	\centering
	\begin{lstlisting}[
        language=Go,
        xleftmargin=.08\textwidth,
        xrightmargin=.08\textwidth
    ]
func (e *Env) Get(key string) (Object, bool) {
    obj, ok := e.store[key]
    if !ok && e.outer != nil {
        obj, ok = e.outer.Get(key)
    }
    return obj, ok
}
func (e *Env) Set(key string, val Object) Object {
    e.store[key] = val
    return val
}    
\end{lstlisting}
	\caption{Код функций записи и получения значений окружения}
	\label{f:code_getsetEnv}
\end{figure}

\clearpage


Использование единого хранилища значений переменных для всей области видимости программы вносит некоторые ограничения.
Например в аргументах функции могут быть определены параметры, имена которых совпадают с объявленным ранее переменным.
Это некорректное поведение, так как первое объявленное значение будет перезаписано другим, переданным в функцию.
Пример кода такой ситуации приведен на рисунке~\ref{f:code_rewriteVarError}.

\begin{figure}[!htb]
	\centering
    \vspace{\toppaddingoffigure}
	\begin{lstlisting}[
        language=Go,
        xleftmargin=.08\textwidth,
        xrightmargin=.08\textwidth
    ]
x = 10;
f = func(x) {
    return x * 10;
}
f(5);
x; //5
\end{lstlisting}
	\caption{Пример кода некорректного поведения}
	\label{f:code_rewriteVarError}
\end{figure}

В данном коде, имя параметра функции совпадает с именем переменной -- x.
Переменная <<x>> объявлена со значением, равным 10.
Данное значение перезапишется значением, переданным в функцию при ее вызове, в данном случае значением, равным 5.
Таким образом, после вызова функции значение переменной <<x>> будет неявно изменено и станет равным 5.
Данная логика работы является ошибочной.

Решение лежит в выделении внутреннего окружения функции при её вызове.
Именно для такого случая в ранее рассмотренной структуре Env содержится ссылка на другой экземпляр структуры такого же типа.
Так, при вызове функции, необходимо создать новый экземпляр окружения, записать в него переданные аргументы и установить ссылку на внешнее окружение,
то из которого была вызвана функция. Такой подход позволит корректно выполнять вложенные функции и рекурсивные вызовы.

Для строк и массивов реализованы несколько встроенных функций:

\begin{itemize}
    \item len -- определение длины строки или массива;
    \item push -- добавление элемента в конец массива;
    \item first -- получение первого элемента массива;
    \item last -- получение последнего элемента массива.
\end{itemize}
% Код реализации встроенных функций приведен в приложении Г.

Хэш-карты реализованы на базе хэш-карт языка Go.
В качестве ключа могут быть следующие типы данных: строка, булево значение, число.
Так как данные типы представлены в виде внутренних объектов, нельзя брать тип Object в качестве ключа.
При занесении значения в хэш-карту и попытке последующего его получения ключи будут представлять разные экземпляры несмотря на одинаковое значение.
На рисунке~\ref{f:code_hashmapGetValueExample} приведен наглядный пример.

\begin{figure}[!htb]
	\centering
    \vspace{\toppaddingoffigure}
	\begin{lstlisting}[
        language=Go
    ]
X = { "name": "Bob" }
X["name"]     
\end{lstlisting}
	\caption{Пример получения значения хэш-карты}
	\label{f:code_hashmapGetValueExample}
\end{figure}

Строковой ключ «name» при попытке получить значение из карты не будет возвращать значение "Bob",
так как при вычислении выражения будет создан новый экземпляр объекта, представляющего строку.
Для решения этой проблемы можно использовать в качестве ключа строку, содержащую хэш от значения объекта.

Фрагмент кода объектной системы с реализацией хэш-карт представлен на рисунке~\ref{f:code_hashmapObject}.

Фрагмент кода функции вычисления целочисленного инфиксного выражения приведен на рисунке~\ref{f:code_evalIntInfixExpr}.

\clearpage

\begin{figure}[!htb]
	\centering
	\begin{lstlisting}[
        language=Go,
        xleftmargin=.08\textwidth,
        xrightmargin=.08\textwidth
    ]
type HashKey struct {
    Type  ObjectType
    Value uint64
}

type Hashable interface {
    HashKey() HashKey
}

type HashPair struct {
    Key   Object
    Value Object
}

type HashMap struct {
    Pairs map[HashKey]HashPair
}

func (h *HashMap) Type() ObjectType { return HASH_MAP_OBJ }
func (h *HashMap) ToString() string {
    var out bytes.Buffer

    pairs := []string{}
    for _, el := range h.Pairs {
        pairs = append(pairs, fmt.Sprintf("%s: %s", el.Key.ToString(), el.Value.ToString()))
    }

    out.WriteString("[")
    out.WriteString(strings.Join(pairs, ", "))
    out.WriteString("]")

    return out.String()
} 
\end{lstlisting}
	\caption{Фрагмент кода реализации хэш-карт}
	\label{f:code_hashmapObject}
\end{figure}

\clearpage

\begin{figure}[!htb]
	\centering
	\begin{lstlisting}[
        language=Go,
        xleftmargin=.08\textwidth,
        xrightmargin=.08\textwidth
    ]
func evalIntInfixExpr(op string, left object.Object, right object.Object) object.Object {
    lVal := left.(*object.Integer).Value
    rVal := right.(*object.Integer).Value
    switch op {
    case "+":
        return &object.Integer{Value: lVal + rVal}
    case "-":
        return &object.Integer{Value: lVal - rVal}
    case "*":
        return &object.Integer{Value: lVal * rVal}
    case "/":
        return &object.Integer{Value: lVal / rVal}
    case "%":
        return &object.Integer{Value: lVal % rVal}
    case "==":
        return boolToBooleanObj(lVal == rVal)
    case "!=":
        return boolToBooleanObj(lVal != rVal)
    case "<":
        return boolToBooleanObj(lVal < rVal)
    case ">":
        return boolToBooleanObj(lVal > rVal)
    case "<=":
        return boolToBooleanObj(lVal <= rVal)
    case ">=":
        return boolToBooleanObj(lVal >= rVal)
    default:
        return newError("unknown operator: %s %s %s", left.Type(), op, right.Type())
    }
}
\end{lstlisting}
	\caption{Фрагмент кода функции вычисления целочисленного инфиксного выражения}
	\label{f:code_evalIntInfixExpr}
\end{figure}

% Фрагменты кода исполнителя приведены в приложении В.

Пример успешного выполнения программы для указанных входных данных представлен на рисунке~\ref{f:evalSuccessExample}.

Входная строка: $5 + 1 * 20 / (5 + 5)$.

\clearpage

\begin{figure}[!htb]
	\centering
	\includegraphics[width=0.4\textwidth]{evaluator/evalSuccessExample.png}
	\caption{Пример успешного выполнения программы}
	\label{f:evalSuccessExample}
\end{figure}

Пример завершения работы программы с семантической ошибкой для указанных ниже входных данных показан на рисунке~\ref{f:evalErrorExample}.

Входная строка представлена на рисунке~\ref{f:code_evalErrorExample}.

\begin{figure}[!htb]
	\centering
    \vspace{\toppaddingoffigure}
	\begin{lstlisting}[
        language=Go,
        xleftmargin=.08\textwidth,
        xrightmargin=.08\textwidth
    ]
x = 5 + 1 * 20 / (5 + 5)
if (x > true) {
    return 1
}    
\end{lstlisting}
	\caption{Пример кода, содержащего ошибку}
	\label{f:code_evalErrorExample}
\end{figure}

\begin{figure}[!htb]
	\centering
	\includegraphics[width=0.8\textwidth]{evaluator/evalErrorExample.png}
	\caption{Пример завершения работы программы с ошибкой}
	\label{f:evalErrorExample}
\end{figure}

\pagebreak
\section{Тестирование и экспериментальная апробация}

В данном разделе представлены этапы тестирования и экспериментальной апробации разработанного программного продукта.

\subsection{Тестирование}

Тестирование является одним из ключевых этапов разработки, направленным на подтверждение корректности работы программы.

Выделяют три основных вида тестирования \refref{ref:testing}:
\begin{itemize}
    \item модульное тестирование;
    \item интеграционное тестирование;
    \item функциональное тестирование.
\end{itemize}

Модульное тестирование или юнит-тестирование позволяет проверить отдельные изолированные компоненты системы.
Оно направлено на проверку правильности функционирования каждого блока с помощью входных данных и подтверждения соответствия результата теста ожидаемым результатам.

Интеграционное тестирование направлено на проверку корректности взаимодействия нескольких модулей как единой группы.
Интеграционные тесты позволяют выявить проблемы, связанные с зависимостями и обменом информацией между компонентами системы.

Функциональное тестирование -- тип тестирования, который направлен на проверку соответствия работы программы заданной функциональности.
Основная цель -- убедиться, что разработанное программное обеспечение работает правильно и обеспечивает требуемую функциональность.

Проверка корректности работы лексера, парсера, семантического анализатора и исполнителя выполнена с помощью модульных тестов.
Написаны тесты для большинства основных функций.
Некоторые тест-кейсы приведены в таблицах~\ref{t:testCases_infixIntExpr}-13.

Для реализации и запуска тестов в разработанной программе использовалась встроенная среду выполнения Go утилита <<go test>>.
Эта утилита предоставляет удобный и мощный инструмент для создания и выполнения тестов, включая модульные и функциональные тесты.
<<Go test>> позволяет одной командой запускать все реализованные тест-кейсы в проекте и получать результат их выполнения \refref{ref:go-testing}.

В ходе запуска тестирования все тесты выполнились успешно. Результат тестирования представлен на рисунке~\ref{f:testResult}.

\begin{figure}[ht]
    \centering
    \vspace{\toppaddingoffigure}
    \includegraphics[width=0.9\textwidth]{evaluator/testResult.png}
    \caption{Результаты запуска тестов}
    \label{f:testResult}
\end{figure}

\begin{table}[!ht]
    \Large
    \centering
    \begin{threeparttable}
        \caption{Тест-кейсы исполнения целочисленного выражения}
        \label{t:testCases_infixIntExpr}
        \begin{tabularx}{\textwidth}{|X|c|}
            \hline
            \multicolumn{1}{|>{\centering\arraybackslash}X|}{Входные данные} & Ожидаемый результат \\
            \hline
            4                                                                & 4                   \\
            \hline
            -5                                                               & -5                  \\
            \hline
            2 + 2                                                            & 4                   \\
            \hline
            1 + 2 + 3 + 4 + 5 - 1 - 2 - 3                                    & 9                   \\
            \hline
            2 * 3 * 4 * 5 * 6 * 7 * 8 * 9                                    & 362880              \\
            \hline
            10 + 10 * 2                                                      & 30                  \\
            \hline
            (10 + 10) * 2                                                    & 40                  \\
            \hline
            100 / 2 * 2 + 5                                                  & 105                 \\
            \hline
            100 / (2 * 2) - 200                                              & -175                \\
            \hline
            5 \% 2                                                           & 1                   \\
            \hline
            4 \% 2                                                           & 0                   \\
            \hline
        \end{tabularx}
    \end{threeparttable}
    \vspace{\bottompaddingoftable}
\end{table}

\clearpage

\begin{table}[!ht]
    \Large
    \centering
    \begin{threeparttable}
        \caption{Тест-кейсы исполнения встроенных функций}
        \label{t:testCases_builtins}
        \begin{tabularx}{\textwidth}{|X|c|}
            \hline
            \multicolumn{1}{|>{\centering\arraybackslash}X|}{Входные данные} & Ожидаемый результат                        \\
            \hline
            len("abc")                                                       & 3                                          \\
            \hline
            len("abc" + "efg")                                               & 6                                          \\
            \hline
            len("")                                                          & 0                                          \\
            \hline
            len(1)                                                           & type of argument not supported: INTEGER    \\
            \hline
            len("a", "b")                                                    & wrong number of arguments: 2 want: 1       \\
            \hline
            x = "abc"; len(x)                                                & 3                                          \\
            \hline
            len({[}{]})                                                      & 0                                          \\
            \hline
            x = {[}1, 2, 3{]}; len(x)                                        & 3                                          \\
            \hline
            push({[}{]}, 4)                                                  & {[}4{]}                                    \\
            \hline
            push({[}1, 2, 3{]}, 4)                                           & {[}1, 2, 3, 4{]}                           \\
            \hline
            push("a", 4)                                                     & first argument must be ARRAY, got: STRING" \\
            \hline
            first({[}{]})                                                    & null                                       \\
            \hline
            first({[}1{]})                                                   & 1                                          \\
            \hline
            first({[}3, 2, 1{]})                                             & 3                                          \\
            \hline
            first("a")                                                       & argument must be ARRAY, got: STRING        \\
            \hline
            last({[}{]})                                                     & null                                       \\
            \hline
            last({[}1{]})                                                    & 1                                          \\
            \hline
            intToString("a")  & argument must be INTEGER, got: STRING \\
            \hline
            intToString(123)  & 123                                   \\
            \hline
            intToString(1, 2) & wrong number of arguments: 2 want: 1  \\
            \hline
        \end{tabularx}
    \end{threeparttable}
    \vspace{\bottompaddingoftable}
\end{table}

\begin{table}[!ht]
    \Large
    \centering
    \begin{threeparttable}
        \caption{Тест-кейсы исполнения условного выражения}
        \label{t:testCases_conditionExpr}
        \begin{tabularx}{\textwidth}{|X|c|}
            \hline
            \multicolumn{1}{|>{\centering\arraybackslash}X|}{Входные данные} & Ожидаемый результат \\
            \hline
            if (true) \{ 50 \}                                               & 50                  \\
            \hline
            if (false) \{ 50 \}                                              & null                \\
            \hline
            if (!false) \{ 50 \}                                             & 50                  \\
            \hline
            if (1 \textless 2) \{ 50 \} else \{ 100 \}                       & 50                  \\
            \hline
            if (1 \textgreater 2) \{ 50 \} else \{ 100 \}                    & 100                 \\
            \hline
            if (true || false) \{ 50 \} else \{ 100 \}                       & 50                  \\
            \hline
            if (true   \&\& false) \{ 50 \} else \{ 100 \}                   & 100                 \\
            \hline
        \end{tabularx}
    \end{threeparttable}
    \vspace{\bottompaddingoftable}
\end{table}

% \clearpage

\begin{table}[!ht]
    \Large
    \centering
    \begin{threeparttable}
        \caption{Тест-кейсы семантических ошибок}
        \label{t:testCases_semanticErrors}
        \begin{tabularx}{\textwidth}{|X|c|}
            \hline
            \multicolumn{1}{|>{\centering\arraybackslash}X|}{Входные данные} & Ожидаемый результат                   \\
            \hline
            true + false                                                     & unknown operator: BOOLEAN + BOOLEAN   \\
            \hline
            1; true - false; 2                                               & unknown operator: BOOLEAN - BOOLEAN   \\
            \hline
            1; true + false + true + true; 2                                 & unknown operator: BOOLEAN + BOOLEAN   \\
            \hline
            -true                                                            & unknown operator: -BOOLEAN            \\
            \hline
            true + 3                                                         & type mismatch: BOOLEAN + INTEGER      \\
            \hline
            3 * false                                                        & type mismatch: INTEGER * BOOLEAN      \\
            \hline
            "Hello" * 3                                                      & type mismatch: STRING * INTEGER       \\
            \hline
            "Hello" * "Earth"                                                & unknown operator: STRING * STRING     \\
            \hline
            if (3) \{ 1 \}                                                   & non boolean condition in if statement \\
            \hline
            x = 10; q                                                        & identifier not found: q               \\
            \hline
            true{[}1{]}                                                      & index operator not supported: BOOLEAN \\
            \hline
            123{[}123{]}                                                     & index operator not supported: INTEGER \\
            \hline
        \end{tabularx}
    \end{threeparttable}
    \vspace{\bottompaddingoftable}
\end{table}

\begin{table}[!ht]
    \Large
    \centering
    \begin{threeparttable}
        \caption{Тест-кейсы исполнения вызова функции}
        \label{t:testCases_fnCall}
        \begin{tabularx}{\textwidth}{|X|c|}
            \hline
            \multicolumn{1}{|>{\centering\arraybackslash}X|}{Входные данные} & Ожидаемый результат \\
            \hline
            x = fn(x)\{ x \}; x(10);                                         & 10                  \\
            \hline
            x = fn(x, y)\{ return x * y \}; x(10, 9);                        & 90                  \\
            \hline
            x = fn(x, y, z)\{ return x * (y - z) \}; x(10,   9, 1);          & 80                  \\
            \hline
            x = fn(x, y)\{ return x + y \}; x(10, x(x(1, 1), x(3, 5)));      & 20                  \\
            \hline
            fn(x)\{ x \}(5)                                                  & 5                   \\
            \hline
            c = fn(x)\{                                                      & 9                   \\
            fn(y) \{ x + y \}                                                &                     \\
            \}                                                               &                     \\
            a = c(5);                                                        &                     \\
            a(4)                                                             &                     \\
            \hline
        \end{tabularx}
    \end{threeparttable}
    \vspace{\bottompaddingoftable}
\end{table}


\begin{table}[!ht]
    \Large
    \centering
    \begin{threeparttable}
        \caption{Тест-кейсы исполнения массива}
        \label{t:testCases_arrayExprt}
        \begin{tabularx}{\textwidth}{|X|c|}
            \hline
            \multicolumn{1}{|>{\centering\arraybackslash}X|}{Входные данные} & Ожидаемый результат     \\
            \hline
            {[}1, 2, -33, 5+5, 1 + 2 + 3 + 4 * 5{]}                          & {[}1, 2, -33, 10, 26{]} \\
            \hline
        \end{tabularx}
    \end{threeparttable}
    \vspace{\bottompaddingoftable}
\end{table}

\clearpage

\begin{table}[!ht]
    \Large
    \centering
    \begin{threeparttable}
        \caption{Тест-кейсы исполнения строкового выражения}
        \label{t:testCases_stringExpr}
        \begin{tabularx}{\textwidth}{|X|c|}
            \hline
            \multicolumn{1}{|>{\centering\arraybackslash}X|}{Входные данные} & Ожидаемый результат \\
            \hline
            "Hello Earth"                                                    & Hello Earth         \\
            \hline
            "Hello" + " " + "Earth"                                          & Hello Earth         \\
            \hline
        \end{tabularx}
    \end{threeparttable}
    \vspace{\bottompaddingoftable}
\end{table}

\begin{table}[!ht]
    \Large
    \centering
    \begin{threeparttable}
        \caption{Тест-кейсы исполнения булева выражения}
        \label{t:testCases_boolExpr}
        \begin{tabularx}{\textwidth}{|X|c|}
            \hline
            \multicolumn{1}{|>{\centering\arraybackslash}X|}{Входные данные} & Ожидаемый результат \\
            \hline
            true                                                             & true                \\
            \hline
            false                                                            & false               \\
            \hline
            1 == 1                                                           & true                \\
            \hline
            1 != 1                                                           & false               \\
            \hline
            1 \textless 2                                                    & true                \\
            \hline
            1 \textgreater 2                                                 & false               \\
            \hline
            1 \textless{}= 2                                                 & true                \\
            \hline
            1 \textless{}= 1                                                 & true                \\
            \hline
            1 \textgreater{}= 2                                              & false               \\
            \hline
            1 \textgreater{}= 1                                              & true                \\
            \hline
            true == true                                                     & true                \\
            \hline
            false == false                                                   & true                \\
            \hline
            true == false                                                    & false               \\
            \hline
            true != false                                                    & true                \\
            \hline
            (true == false) == false                                         & true                \\
            \hline
            (1 == 1) == true                                                 & true                \\
            \hline
            (1 \textless{}= 1) == false                                      & false               \\
            \hline
            true || false                                                    & true                \\
            \hline
            true \&\& false                                                  & false               \\
            \hline
            true \&\& true                                                   & true                \\
            \hline
            false \&\& false                                                 & false               \\
            \hline
            false || false                                                   & false               \\
            \hline
            !true                                                            & false               \\
            \hline
            !false                                                           & true                \\
            \hline
            !!false                                                          & false               \\
            \hline
        \end{tabularx}
    \end{threeparttable}
    \vspace{\bottompaddingoftable}
\end{table}

\clearpage

\begin{table}[!ht]
    \Large
    \centering
    \begin{threeparttable}
        \caption{Тест-кейсы исполнения индексного выражения для массива}
        \label{t:testCases_arrayIndexExpr}
        \begin{tabularx}{\textwidth}{|X|c|}
            \hline
            \multicolumn{1}{|>{\centering\arraybackslash}X|}{Входные данные} & Ожидаемый результат \\
            \hline
            {[}1, 2, 5{]}{[}0{]}                                             & 1                   \\
            \hline
            {[}1, 2, 5{]}{[}2{]}                                             & 5                   \\
            \hline
            {[}1, 2, 5{]}{[}3{]}                                             & null                \\
            \hline
            {[}1, 2, 5{]}{[}-1{]}                                            & null                \\
            \hline
            {[}1, 2, 5{]}{[}1+1{]}                                           & 5                   \\
            \hline
            x = 1; {[}1, 2, 5{]}{[}x{]}                                      & 2                   \\
            \hline
            a = {[}1, 2, 5{]}; a{[}0{]} + a{[}1{]} * a{[}2{]}                & 11                  \\
            \hline
        \end{tabularx}
    \end{threeparttable}
    \vspace{\bottompaddingoftable}
\end{table}

\begin{table}[!ht]
    \Large
    \centering
    \begin{threeparttable}
        \caption{Тест-кейсы исполнения индексного выражения для хэш-карты}
        \label{t:testCases_HashMapIndexExpr}
        \begin{tabularx}{\textwidth}{|X|c|}
            \hline
            \multicolumn{1}{|>{\centering\arraybackslash}X|}{Входные данные} & Ожидаемый результат \\
            \hline
            \{"x": 1\}{[}"x"{]}                                              & 1                   \\
            \hline
            \{"x": 1\}{[}"y"{]}                                              & null                \\
            \hline
            \{\}{[}"y"{]}                                                    & null                \\
            \hline
            \{5: 2\}{[}5{]}                                                  & 2                   \\
            \hline
            \{true: 5\}{[}true{]}                                            & 5                   \\
            \hline
        \end{tabularx}
    \end{threeparttable}
    \vspace{\bottompaddingoftable}
\end{table}

\subsection{Экспериментальная апробация}

Экспериментальная апробация направлена на демонстрацию возможностей разработанного предметно-ориентированного языка.

Рассмотрим пример решения некоторой задачи с помощью DSL.
Пусть необходимо реализовать простую корзину товаров с возможностью многоразового выбора, просмотра списка выбранных товаров и их суммарной стоимости.


Полная сконструированная структура бота представлена на рисунке~\ref{f:experimentFull}.
Далее рассмотрим детально этапы построения данной структуры бота.

\clearpage

\begin{figure}[!ht]
	\centering
	\includegraphics[width=1.0\textwidth]{testing/full.png}
	\caption{Полная компонентная структура бота}
	\label{f:experimentFull}
\end{figure}

Первоначально необходимо построить в визуальном конструкторе цепочку блоков, определяющую пользовательскую корзину и стоимость товара.
Для простоты список товаров будет состоять из двух позиций.
Цепочка, определяющая корзину и стоимость товара представлена на рисунке~\ref{f:experimentCartPrice}.
Код с объявлением стоимости единицы товара приведен на рисунке~\ref{f:experimentCodePrice}.

\begin{figure}[!ht]
	\centering
	\vspace{\toppaddingoffigure}
	\includegraphics[width=0.7\textwidth]{testing/сartPrice.png}
	\caption{Цепочка, определяющая корзину и стоимость товара}
	\label{f:experimentCartPrice}
\end{figure}

\clearpage

\begin{figure}[!ht]
	\centering
	\includegraphics[width=0.6\textwidth]{testing/codePrice.png}
	\caption{Пример кода на DSL}
	\label{f:experimentCodePrice}
\end{figure}

Затем необходимо предоставить пользователю бота выбор товара. Для этого отправим сообщение с двумя кнопками,
обозначающими возможные варианты ответа. Если ответ пользователя не будет принадлежать этим возможным вариантам, запрос будет отправлен повторно.
После выбора пользователем определенного товара, увеличим количество данного товара на 1 с помощью компонента <<код>>.
Цепочка выбора товара представлена на рисунке~\ref{f:experimentSelectItem}.

\begin{figure}[!ht]
	\centering
	\vspace{\toppaddingoffigure}
	\includegraphics[width=0.45\textwidth]{testing/selectItem.png}
	\caption{Цепочка выбора товара}
	\label{f:experimentSelectItem}
\end{figure}

Код увеличения количества единиц товара в корзине при его выборе пользователем представлен на рисунке~\ref{f:expCodeAddItem}.

\clearpage

\begin{figure}[!ht]
	\centering
	\includegraphics[width=0.6\textwidth]{testing/expCodeAddItem.png}
	\caption{Код добавления товара в корзину}
	\label{f:expCodeAddItem}
\end{figure}

После того, как пользователь выбрал товар, выполняется отправка сообщения с предложением повторного выбора товара.
В случае отказа выполняется формирование списка товаров из корзины с указанием их стоимости и общей суммы по всем позициям корзины.
Пример кода на предметно-ориентированном языке, который формирует ответ приведен на рисунке~\ref{f:endCode}.

\begin{figure}[!ht]
	\centering
	\vspace{\toppaddingoffigure}
	\begin{lstlisting}[
        language=Go,
		xleftmargin=.08\textwidth,
        xrightmargin=.08\textwidth
    ]
pprice = cart["Платье"] * price["Платье"];
tprice = cart["Туфли"] * price["Туфли"];

t = ""
if(cart["Платье"] > 0) {
	t = t + "- Платье: " + intToString(cart["Платье"]) + " ед.: " + intToString(pprice) + " р. \n"
}

if(cart["Туфли"] > 0) { 
	t = t + "- Туфли: " +  intToString(cart["Туфли"]) + " ед.: " + intToString(tprice) + " р. \n"
}

if(cart["Туфли"] == 0 && cart["Платье"] == 0) {
	t = "Корзина пуста\n"
}

s = "Ваша корзина:\n " + t + "\n Общая стоимость: " + intToString(pprice + tprice) + " р."
s
\end{lstlisting}
	\caption{Код формирования сообщения с корзиной}
	\label{f:endCode}
\end{figure}

\clearpage

Демонстрация работы запущенного Telegram бота представлена на рисунке~\ref{f:experimentDemo}.

\begin{figure}[!ht]
	\centering
	\includegraphics[width=0.6\textwidth]{testing/demo.png}
	\caption{Демонстрация работы бота}
	\label{f:experimentDemo}
\end{figure}

\section*{Вывод}
В данном разделе было проведено тестирование разработанного программного продукта.
На базе встроенной в среду выполнения Go утилиты <<go test>> были реализованы юнит-тесты, охватывающие большинство функций модуля интерпретации.

Также была выполнена экспериментальная апробация, показывающая пример расширения возможностей визуального конструктора Telegram ботов
за счет использования компонента, для запуска кода на разработанном предметно-ориентированном языке.
\clearpage



\section*{Вывод}

В данном разделе на основании разработанных архитектурно-структурных решениях и алгоритмах функционирования выполнена программная реализация модуля интерпретации предметно-ориентированного языка.
С помощью выбранных инструментов разработки выполнена программная реализация лексического анализатора, синтаксического анализатора, семантического анализатора и исполнителя.

Также на базе встроенной в среду выполнения Go утилиты <<go test>> были реализованы юнит-тесты, охватывающие большинство функций модуля интерпретации.
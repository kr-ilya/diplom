\subsection{Реализация лексического анализатора}

Лексический анализатор состоит из следующих основных компонентов:
\begin{itemize}
    \item токены -- определение структуры и типов токенов, представляющих основные элементы языка;
    \item конечный автомат -- принимает на вход поток символов и определяет токены переходя между состояниями в соответствии с правилами языка.
\end{itemize}

Токен представляет собой структуру, содержащую информацию о типе токена и его значение, представленное в виде строки.
Код структуры, представляющей токен приведен на рисунке~\ref{f:code_tokenStruct}.

\begin{figure}[ht]
	\centering
	\vspace{\toppaddingoffigure}
	\begin{lstlisting}[
        language=Go
    ]
type TokenType = string
type Token struct {
    Type    TokenType
    Literal string
}        
\end{lstlisting}
	\caption{Структура, представляющая токен}
	\label{f:code_tokenStruct}
\end{figure}

Список возможных токенов представлен в таблице~\ref{t:tokens}.

Программную реализацию токенов можно выполнить с помощью списка константных значений.
Фрагмент кода реализации токенов представлен на рисунке~\ref{f:code_tokensFragemnt}.

\begin{figure}[ht]
	\centering
	\begin{lstlisting}
IDENT  = "IDENT"  // x, t, add
INT    = "INT"    // 123
STRING = "STRING" // "abcde"
ASSIGN = "="
PLUS   = "+"
STAR   = "*"
LT  = "<"
GT  = ">"
LAND = "&&"
LOR  = "||"

// keywords
IF     = "IF"
ELSE   = "ELSE"
TRUE   = "TRUE"    
\end{lstlisting}
	\caption{Фрагмент кода реализации токенов}
	\label{f:code_tokensFragemnt}
\end{figure}

Лексер представляет собой структуру, содержащую информацию о входной строке кода,
текущем считанном символе, позиции курсора и других технических значениях, необходимых для корректной работы анализатора.
Структура представляющая лексер приведена на рисунке~\ref{f:code_lexerStruct}.

\begin{figure}[ht]
	\centering
	\vspace{\toppaddingoffigure}
	\begin{lstlisting}[
        language=Go,
        xleftmargin=.08\textwidth,
        xrightmargin=.08\textwidth
    ]
type Lexer struct {
    input   string
    ch      byte // current char
    pos     int  // current position (on current char)
    readPos int  // position after current char
    nlsemi  bool // if "true" '\n' translate to ';'
    loPos   token.Pos
}    
\end{lstlisting}
	\caption{Структура, представляющая лексер}
	\label{f:code_lexerStruct}
\end{figure}

Основная функция лексического анализатора может быть реализована в формате конечного автомата.
При получении очередного символа из входной строки кода его необходимо сопоставить с одним из токенов.
Стоит заметить, что некоторые токены формируются за счет двух и более символом, например токен <<LAND>> (\&\&), идентификаторы, ключевые слова и т.д. 
В этом случае, необходимо продолжать получение символов из входной строки до тех пор, пока не будет однозначно определен токен.

Основная функция определения токена выполнена в виде конструкции switch-case.
Фрагмент кода представлен на рисунке~\ref{f:code_lexerFragment}.

\clearpage

\begin{figure}[!ht]
	\centering
    \begin{lstlisting}[
        language=Go,
        xleftmargin=.08\textwidth,
        xrightmargin=.08\textwidth
    ]
func (l *Lexer) NextToken() (token.Token, token.Pos) {
    l.skipWhitespace()
    nlsemi := false
    var tok token.Token
    switch l.ch {
    case '\n':
        tok = newToken(token.SEMICOLON, l.ch)
    case '=':
        if l.peekChar() == '=' {
            l.readChar()
            literal := "=="
            tok = token.Token{Type: token.EQ, Literal: literal}
        } else {
            tok = newToken(token.ASSIGN, l.ch)
        }
    case '+':
        tok = newToken(token.PLUS, l.ch)
    case '-':
        tok = newToken(token.MINUS, l.ch)
    case '*':
        tok = newToken(token.STAR, l.ch)
    case '/':
        tok = newToken(token.SLASH, l.ch)
    case '!':
        if l.peekChar() == '=' {
            l.readChar()
            literal := "!="
            tok = token.Token{Type: token.NEQ, Literal: literal}
        } else {
            tok = newToken(token.EXCLAMINATION, l.ch)
        }
    case '%':
		tok = newToken(token.PERCENT, l.ch)
	case '<':
		if l.peekChar() == '=' {
			l.readChar()
			literal := "<="
			tok = token.Token{Type: token.LEQ, Literal: literal}
		} else {
			tok = newToken(token.LT, l.ch)
		}
\end{lstlisting}
	\caption{Фрагмент кода лексера}
	\label{f:code_lexerFragment}
\end{figure}
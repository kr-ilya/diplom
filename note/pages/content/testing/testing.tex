\section{Тестирование и экспериментальная апробация}

В данном разделе представлены этапы тестирования и экспериментальной апробации разработанного программного продукта.

\subsection{Тестирование}

Тестирование является одним из ключевых этапов разработки, направленным на подтверждение корректности работы программы.

Выделяют три основных вида тестирования \refref{ref:testing}:
\begin{itemize}
    \item модульное тестирование;
    \item интеграционное тестирование;
    \item функциональное тестирование.
\end{itemize}

Модульное тестирование или юнит-тестирование позволяет проверить отдельные изолированные компоненты системы.
Оно направлено на проверку правильности функционирования каждого блока с помощью входных данных и подтверждения соответствия результата теста ожидаемым результатам.

Интеграционное тестирование направлено на проверку корректности взаимодействия нескольких модулей как единой группы.
Интеграционные тесты позволяют выявить проблемы, связанные с зависимостями и обменом информацией между компонентами системы.

Функциональное тестирование -- тип тестирования, который направлен на проверку соответствия работы программы заданной функциональности.
Основная цель -- убедиться, что разработанное программное обеспечение работает правильно и обеспечивает требуемую функциональность.

Проверка корректности работы лексера, парсера, семантического анализатора и исполнителя выполнена с помощью модульных тестов.
Написаны тесты для большинства основных функций.
Некоторые тест-кейсы приведены в таблицах~\ref{t:testCases_infixIntExpr}-13.

Для реализации и запуска тестов в разработанной программе использовалась встроенная среду выполнения Go утилита <<go test>>.
Эта утилита предоставляет удобный и мощный инструмент для создания и выполнения тестов, включая модульные и функциональные тесты.
<<Go test>> позволяет одной командой запускать все реализованные тест-кейсы в проекте и получать результат их выполнения \refref{ref:go-testing}.

В ходе запуска тестирования все тесты выполнились успешно. Результат тестирования представлен на рисунке~\ref{f:testResult}.

\begin{figure}[ht]
    \centering
    \vspace{\toppaddingoffigure}
    \includegraphics[width=0.9\textwidth]{evaluator/testResult.png}
    \caption{Результаты запуска тестов}
    \label{f:testResult}
\end{figure}

\begin{table}[!ht]
    \Large
    \centering
    \begin{threeparttable}
        \caption{Тест-кейсы исполнения целочисленного выражения}
        \label{t:testCases_infixIntExpr}
        \begin{tabularx}{\textwidth}{|X|c|}
            \hline
            \multicolumn{1}{|>{\centering\arraybackslash}X|}{Входные данные} & Ожидаемый результат \\
            \hline
            4                                                                & 4                   \\
            \hline
            -5                                                               & -5                  \\
            \hline
            2 + 2                                                            & 4                   \\
            \hline
            1 + 2 + 3 + 4 + 5 - 1 - 2 - 3                                    & 9                   \\
            \hline
            2 * 3 * 4 * 5 * 6 * 7 * 8 * 9                                    & 362880              \\
            \hline
            10 + 10 * 2                                                      & 30                  \\
            \hline
            (10 + 10) * 2                                                    & 40                  \\
            \hline
            100 / 2 * 2 + 5                                                  & 105                 \\
            \hline
            100 / (2 * 2) - 200                                              & -175                \\
            \hline
            5 \% 2                                                           & 1                   \\
            \hline
            4 \% 2                                                           & 0                   \\
            \hline
        \end{tabularx}
    \end{threeparttable}
    \vspace{\bottompaddingoftable}
\end{table}

\clearpage

\begin{table}[!ht]
    \Large
    \centering
    \begin{threeparttable}
        \caption{Тест-кейсы исполнения встроенных функций}
        \label{t:testCases_builtins}
        \begin{tabularx}{\textwidth}{|X|c|}
            \hline
            \multicolumn{1}{|>{\centering\arraybackslash}X|}{Входные данные} & Ожидаемый результат                        \\
            \hline
            len("abc")                                                       & 3                                          \\
            \hline
            len("abc" + "efg")                                               & 6                                          \\
            \hline
            len("")                                                          & 0                                          \\
            \hline
            len(1)                                                           & type of argument not supported: INTEGER    \\
            \hline
            len("a", "b")                                                    & wrong number of arguments: 2 want: 1       \\
            \hline
            x = "abc"; len(x)                                                & 3                                          \\
            \hline
            len({[}{]})                                                      & 0                                          \\
            \hline
            x = {[}1, 2, 3{]}; len(x)                                        & 3                                          \\
            \hline
            push({[}{]}, 4)                                                  & {[}4{]}                                    \\
            \hline
            push({[}1, 2, 3{]}, 4)                                           & {[}1, 2, 3, 4{]}                           \\
            \hline
            push("a", 4)                                                     & first argument must be ARRAY, got: STRING" \\
            \hline
            first({[}{]})                                                    & null                                       \\
            \hline
            first({[}1{]})                                                   & 1                                          \\
            \hline
            first({[}3, 2, 1{]})                                             & 3                                          \\
            \hline
            first("a")                                                       & argument must be ARRAY, got: STRING        \\
            \hline
            last({[}{]})                                                     & null                                       \\
            \hline
            last({[}1{]})                                                    & 1                                          \\
            \hline
            intToString("a")  & argument must be INTEGER, got: STRING \\
            \hline
            intToString(123)  & 123                                   \\
            \hline
            intToString(1, 2) & wrong number of arguments: 2 want: 1  \\
            \hline
        \end{tabularx}
    \end{threeparttable}
    \vspace{\bottompaddingoftable}
\end{table}

\begin{table}[!ht]
    \Large
    \centering
    \begin{threeparttable}
        \caption{Тест-кейсы исполнения условного выражения}
        \label{t:testCases_conditionExpr}
        \begin{tabularx}{\textwidth}{|X|c|}
            \hline
            \multicolumn{1}{|>{\centering\arraybackslash}X|}{Входные данные} & Ожидаемый результат \\
            \hline
            if (true) \{ 50 \}                                               & 50                  \\
            \hline
            if (false) \{ 50 \}                                              & null                \\
            \hline
            if (!false) \{ 50 \}                                             & 50                  \\
            \hline
            if (1 \textless 2) \{ 50 \} else \{ 100 \}                       & 50                  \\
            \hline
            if (1 \textgreater 2) \{ 50 \} else \{ 100 \}                    & 100                 \\
            \hline
            if (true || false) \{ 50 \} else \{ 100 \}                       & 50                  \\
            \hline
            if (true   \&\& false) \{ 50 \} else \{ 100 \}                   & 100                 \\
            \hline
        \end{tabularx}
    \end{threeparttable}
    \vspace{\bottompaddingoftable}
\end{table}

% \clearpage

\begin{table}[!ht]
    \Large
    \centering
    \begin{threeparttable}
        \caption{Тест-кейсы семантических ошибок}
        \label{t:testCases_semanticErrors}
        \begin{tabularx}{\textwidth}{|X|c|}
            \hline
            \multicolumn{1}{|>{\centering\arraybackslash}X|}{Входные данные} & Ожидаемый результат                   \\
            \hline
            true + false                                                     & unknown operator: BOOLEAN + BOOLEAN   \\
            \hline
            1; true - false; 2                                               & unknown operator: BOOLEAN - BOOLEAN   \\
            \hline
            1; true + false + true + true; 2                                 & unknown operator: BOOLEAN + BOOLEAN   \\
            \hline
            -true                                                            & unknown operator: -BOOLEAN            \\
            \hline
            true + 3                                                         & type mismatch: BOOLEAN + INTEGER      \\
            \hline
            3 * false                                                        & type mismatch: INTEGER * BOOLEAN      \\
            \hline
            "Hello" * 3                                                      & type mismatch: STRING * INTEGER       \\
            \hline
            "Hello" * "Earth"                                                & unknown operator: STRING * STRING     \\
            \hline
            if (3) \{ 1 \}                                                   & non boolean condition in if statement \\
            \hline
            x = 10; q                                                        & identifier not found: q               \\
            \hline
            true{[}1{]}                                                      & index operator not supported: BOOLEAN \\
            \hline
            123{[}123{]}                                                     & index operator not supported: INTEGER \\
            \hline
        \end{tabularx}
    \end{threeparttable}
    \vspace{\bottompaddingoftable}
\end{table}

\begin{table}[!ht]
    \Large
    \centering
    \begin{threeparttable}
        \caption{Тест-кейсы исполнения вызова функции}
        \label{t:testCases_fnCall}
        \begin{tabularx}{\textwidth}{|X|c|}
            \hline
            \multicolumn{1}{|>{\centering\arraybackslash}X|}{Входные данные} & Ожидаемый результат \\
            \hline
            x = fn(x)\{ x \}; x(10);                                         & 10                  \\
            \hline
            x = fn(x, y)\{ return x * y \}; x(10, 9);                        & 90                  \\
            \hline
            x = fn(x, y, z)\{ return x * (y - z) \}; x(10,   9, 1);          & 80                  \\
            \hline
            x = fn(x, y)\{ return x + y \}; x(10, x(x(1, 1), x(3, 5)));      & 20                  \\
            \hline
            fn(x)\{ x \}(5)                                                  & 5                   \\
            \hline
            c = fn(x)\{                                                      & 9                   \\
            fn(y) \{ x + y \}                                                &                     \\
            \}                                                               &                     \\
            a = c(5);                                                        &                     \\
            a(4)                                                             &                     \\
            \hline
        \end{tabularx}
    \end{threeparttable}
    \vspace{\bottompaddingoftable}
\end{table}


\begin{table}[!ht]
    \Large
    \centering
    \begin{threeparttable}
        \caption{Тест-кейсы исполнения массива}
        \label{t:testCases_arrayExprt}
        \begin{tabularx}{\textwidth}{|X|c|}
            \hline
            \multicolumn{1}{|>{\centering\arraybackslash}X|}{Входные данные} & Ожидаемый результат     \\
            \hline
            {[}1, 2, -33, 5+5, 1 + 2 + 3 + 4 * 5{]}                          & {[}1, 2, -33, 10, 26{]} \\
            \hline
        \end{tabularx}
    \end{threeparttable}
    \vspace{\bottompaddingoftable}
\end{table}

\clearpage

\begin{table}[!ht]
    \Large
    \centering
    \begin{threeparttable}
        \caption{Тест-кейсы исполнения строкового выражения}
        \label{t:testCases_stringExpr}
        \begin{tabularx}{\textwidth}{|X|c|}
            \hline
            \multicolumn{1}{|>{\centering\arraybackslash}X|}{Входные данные} & Ожидаемый результат \\
            \hline
            "Hello Earth"                                                    & Hello Earth         \\
            \hline
            "Hello" + " " + "Earth"                                          & Hello Earth         \\
            \hline
        \end{tabularx}
    \end{threeparttable}
    \vspace{\bottompaddingoftable}
\end{table}

\begin{table}[!ht]
    \Large
    \centering
    \begin{threeparttable}
        \caption{Тест-кейсы исполнения булева выражения}
        \label{t:testCases_boolExpr}
        \begin{tabularx}{\textwidth}{|X|c|}
            \hline
            \multicolumn{1}{|>{\centering\arraybackslash}X|}{Входные данные} & Ожидаемый результат \\
            \hline
            true                                                             & true                \\
            \hline
            false                                                            & false               \\
            \hline
            1 == 1                                                           & true                \\
            \hline
            1 != 1                                                           & false               \\
            \hline
            1 \textless 2                                                    & true                \\
            \hline
            1 \textgreater 2                                                 & false               \\
            \hline
            1 \textless{}= 2                                                 & true                \\
            \hline
            1 \textless{}= 1                                                 & true                \\
            \hline
            1 \textgreater{}= 2                                              & false               \\
            \hline
            1 \textgreater{}= 1                                              & true                \\
            \hline
            true == true                                                     & true                \\
            \hline
            false == false                                                   & true                \\
            \hline
            true == false                                                    & false               \\
            \hline
            true != false                                                    & true                \\
            \hline
            (true == false) == false                                         & true                \\
            \hline
            (1 == 1) == true                                                 & true                \\
            \hline
            (1 \textless{}= 1) == false                                      & false               \\
            \hline
            true || false                                                    & true                \\
            \hline
            true \&\& false                                                  & false               \\
            \hline
            true \&\& true                                                   & true                \\
            \hline
            false \&\& false                                                 & false               \\
            \hline
            false || false                                                   & false               \\
            \hline
            !true                                                            & false               \\
            \hline
            !false                                                           & true                \\
            \hline
            !!false                                                          & false               \\
            \hline
        \end{tabularx}
    \end{threeparttable}
    \vspace{\bottompaddingoftable}
\end{table}

\clearpage

\begin{table}[!ht]
    \Large
    \centering
    \begin{threeparttable}
        \caption{Тест-кейсы исполнения индексного выражения для массива}
        \label{t:testCases_arrayIndexExpr}
        \begin{tabularx}{\textwidth}{|X|c|}
            \hline
            \multicolumn{1}{|>{\centering\arraybackslash}X|}{Входные данные} & Ожидаемый результат \\
            \hline
            {[}1, 2, 5{]}{[}0{]}                                             & 1                   \\
            \hline
            {[}1, 2, 5{]}{[}2{]}                                             & 5                   \\
            \hline
            {[}1, 2, 5{]}{[}3{]}                                             & null                \\
            \hline
            {[}1, 2, 5{]}{[}-1{]}                                            & null                \\
            \hline
            {[}1, 2, 5{]}{[}1+1{]}                                           & 5                   \\
            \hline
            x = 1; {[}1, 2, 5{]}{[}x{]}                                      & 2                   \\
            \hline
            a = {[}1, 2, 5{]}; a{[}0{]} + a{[}1{]} * a{[}2{]}                & 11                  \\
            \hline
        \end{tabularx}
    \end{threeparttable}
    \vspace{\bottompaddingoftable}
\end{table}

\begin{table}[!ht]
    \Large
    \centering
    \begin{threeparttable}
        \caption{Тест-кейсы исполнения индексного выражения для хэш-карты}
        \label{t:testCases_HashMapIndexExpr}
        \begin{tabularx}{\textwidth}{|X|c|}
            \hline
            \multicolumn{1}{|>{\centering\arraybackslash}X|}{Входные данные} & Ожидаемый результат \\
            \hline
            \{"x": 1\}{[}"x"{]}                                              & 1                   \\
            \hline
            \{"x": 1\}{[}"y"{]}                                              & null                \\
            \hline
            \{\}{[}"y"{]}                                                    & null                \\
            \hline
            \{5: 2\}{[}5{]}                                                  & 2                   \\
            \hline
            \{true: 5\}{[}true{]}                                            & 5                   \\
            \hline
        \end{tabularx}
    \end{threeparttable}
    \vspace{\bottompaddingoftable}
\end{table}

\subsection{Экспериментальная апробация}

Экспериментальная апробация направлена на демонстрацию возможностей разработанного предметно-ориентированного языка.

Рассмотрим пример решения некоторой задачи с помощью DSL.
Пусть необходимо реализовать простую корзину товаров с возможностью многоразового выбора, просмотра списка выбранных товаров и их суммарной стоимости.


Полная сконструированная структура бота представлена на рисунке~\ref{f:experimentFull}.
Далее рассмотрим детально этапы построения данной структуры бота.

\clearpage

\begin{figure}[!ht]
	\centering
	\includegraphics[width=1.0\textwidth]{testing/full.png}
	\caption{Полная компонентная структура бота}
	\label{f:experimentFull}
\end{figure}

Первоначально необходимо построить в визуальном конструкторе цепочку блоков, определяющую пользовательскую корзину и стоимость товара.
Для простоты список товаров будет состоять из двух позиций.
Цепочка, определяющая корзину и стоимость товара представлена на рисунке~\ref{f:experimentCartPrice}.
Код с объявлением стоимости единицы товара приведен на рисунке~\ref{f:experimentCodePrice}.

\begin{figure}[!ht]
	\centering
	\vspace{\toppaddingoffigure}
	\includegraphics[width=0.7\textwidth]{testing/сartPrice.png}
	\caption{Цепочка, определяющая корзину и стоимость товара}
	\label{f:experimentCartPrice}
\end{figure}

\clearpage

\begin{figure}[!ht]
	\centering
	\includegraphics[width=0.6\textwidth]{testing/codePrice.png}
	\caption{Пример кода на DSL}
	\label{f:experimentCodePrice}
\end{figure}

Затем необходимо предоставить пользователю бота выбор товара. Для этого отправим сообщение с двумя кнопками,
обозначающими возможные варианты ответа. Если ответ пользователя не будет принадлежать этим возможным вариантам, запрос будет отправлен повторно.
После выбора пользователем определенного товара, увеличим количество данного товара на 1 с помощью компонента <<код>>.
Цепочка выбора товара представлена на рисунке~\ref{f:experimentSelectItem}.

\begin{figure}[!ht]
	\centering
	\vspace{\toppaddingoffigure}
	\includegraphics[width=0.45\textwidth]{testing/selectItem.png}
	\caption{Цепочка выбора товара}
	\label{f:experimentSelectItem}
\end{figure}

Код увеличения количества единиц товара в корзине при его выборе пользователем представлен на рисунке~\ref{f:expCodeAddItem}.

\clearpage

\begin{figure}[!ht]
	\centering
	\includegraphics[width=0.6\textwidth]{testing/expCodeAddItem.png}
	\caption{Код добавления товара в корзину}
	\label{f:expCodeAddItem}
\end{figure}

После того, как пользователь выбрал товар, выполняется отправка сообщения с предложением повторного выбора товара.
В случае отказа выполняется формирование списка товаров из корзины с указанием их стоимости и общей суммы по всем позициям корзины.
Пример кода на предметно-ориентированном языке, который формирует ответ приведен на рисунке~\ref{f:endCode}.

\begin{figure}[!ht]
	\centering
	\vspace{\toppaddingoffigure}
	\begin{lstlisting}[
        language=Go,
		xleftmargin=.08\textwidth,
        xrightmargin=.08\textwidth
    ]
pprice = cart["Платье"] * price["Платье"];
tprice = cart["Туфли"] * price["Туфли"];

t = ""
if(cart["Платье"] > 0) {
	t = t + "- Платье: " + intToString(cart["Платье"]) + " ед.: " + intToString(pprice) + " р. \n"
}

if(cart["Туфли"] > 0) { 
	t = t + "- Туфли: " +  intToString(cart["Туфли"]) + " ед.: " + intToString(tprice) + " р. \n"
}

if(cart["Туфли"] == 0 && cart["Платье"] == 0) {
	t = "Корзина пуста\n"
}

s = "Ваша корзина:\n " + t + "\n Общая стоимость: " + intToString(pprice + tprice) + " р."
s
\end{lstlisting}
	\caption{Код формирования сообщения с корзиной}
	\label{f:endCode}
\end{figure}

\clearpage

Демонстрация работы запущенного Telegram бота представлена на рисунке~\ref{f:experimentDemo}.

\begin{figure}[!ht]
	\centering
	\includegraphics[width=0.6\textwidth]{testing/demo.png}
	\caption{Демонстрация работы бота}
	\label{f:experimentDemo}
\end{figure}

\section*{Вывод}
В данном разделе было проведено тестирование разработанного программного продукта.
На базе встроенной в среду выполнения Go утилиты <<go test>> были реализованы юнит-тесты, охватывающие большинство функций модуля интерпретации.

Также была выполнена экспериментальная апробация, показывающая пример расширения возможностей визуального конструктора Telegram ботов
за счет использования компонента, для запуска кода на разработанном предметно-ориентированном языке.
\clearpage


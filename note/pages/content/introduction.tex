
\csection{Введение}

В современном мире стали популярными такие приложения для быстрого и удобного общения как мессенджеры.
Таких приложений достаточно много, но большинство пользователей сети интернет все чаще отдают предпочтение мессенджеру Telegram как наиболее удобному, надежному и функциональному.
У Telegram имеется функционал чат-ботов.
Любой пользователь может создать своего Telegram бота с помощью общедоступного API, который предоставляет методы для управления ботом.
Однако, для создания и управления Telegram ботами требуется определенный уровень технических знаний и навыков программирования, что может быть преградой для многих пользователей.

Конструктор Telegram ботов позволяет широкому кругу пользователей создавать программные продукты с помощью визуального редактора.
Пользователи могут выстраивать логику работы приложения просто перетаскивая визуальные блоки и соединяя их между собой в логические цепочки.
Это значительно упрощает разработку и делает её доступной даже для тех, кто не имеет глубоких знаний в программировании.
Однако функциональные возможности такого подхода к построению ботов ограничены набором доступных компонентов,
из которых стоится структура бота и их зачастую недостаточно для реализации сложных, специфичных программных продуктов.

Расширение возможностей платформы визуального конструктора возможно
за счёт использования предметно-ориентированного языка программирования.
С помощью написания инструкций на данном языке, пользователи  могут гибко задавать логику
работы разрабатываемого бота, тем самым выходить за рамки функионала, предоставляемого стандартными компонентами.

\pagebreak

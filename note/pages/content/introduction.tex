
\csection{Введение}

В современном мире все больше обретают популярность сервисы
визуального конструирования приложений.

Конструктор позволяет широкому кругу пользователей создавать программные продукты с помощью визуального редактора.
Пользователи могут выстраивать логику работы приложения просто перетаскивая визуальные блоки.
Это значительно упрощает разработку и делает её доступной даже для тех, кто не имеет глубоких знаний в программировании.
Однако функциональные возможности такого подхода к построению приложений ограничены и
их зачастую недостаточно для реализации сложных, специфичных программных продуктов.

Расширение возможностей платформы визуального конструктора возможно
за счёт использования предметно-ориентированного языка программирования.
С помощью написания инструкций на данном языке, пользователи  могут гибко задавать логику
работы разрабатываемого приложения.

\pagebreak

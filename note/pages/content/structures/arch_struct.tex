\newpage

\section{Разработка архитектурно-структурных решений}

Выполнение кода предметно-ориентированного языка осуществляется за счет интерпретатора,
который должен быть спроектирован в виде модуля, подключаемого к серверной части конструктора Telegram ботов.

Обобщенная модульная структура серверной части конструктора представлена на рисунке~\ref{f:modules_server_struct}.

\begin{figure}[ht]
	\centering
	\vspace{\toppaddingoffigure}
	\includegraphics[width=0.7\textwidth]{structures/modules_server_struct.pdf}
	\caption{Модульная структура серверной части конструктора}
	\label{f:modules_server_struct}
\end{figure}

Сервис ботов отвечает за управление состоянием бота и редактирование его компонентной структуры.
Сервис ботов зависит от сервиса пользователей, который предоставляет первому методы для авторизации пользователя.
Также сервис ботов зависит от модуля компонентов, который описывает структуры компонентов и реализует их логику выполнения.

Обслуживающий сервис отвечает за логику работы бота. 
Он выполняет обработку запроса к боту от пользователя Telegram.
В соответствии с разработанной компонентной структурой бота, обслуживающий сервис вызывает методы модуля компонентов для запуска их логики выполнения.

Одним из компонентов является компонент выполнения кода на предметно-ориентированном языке.
Отсюда, модуль компонентов зависит от модуля интерпретации предметно-ориентированного языка.
При запуске компонента выполнения DSL кода, выполняется вызов функции выполнения кода из модуля интерпретации.

В данном проекте для возможности интеграции с серверной частью конструктора Telegram ботов необходимо реализовать программный интерфейс,
который бы предоставлял возможность передачи кода и значений внешних переменных, интерпретатору предметно-ориентированного языка.

Работа интерпретатора состоит из последовательного выполнения следующих этапов:
\begin{enumerate}
	\item лексический анализ;
	\item синтаксический анализ;
	\item семантический анализ;
	\item исполнение команд.
\end{enumerate}

Обобщенная структура интерпретатора представлена на рисунке~\ref{f:full_interpreter_struct}.

\begin{figure}[ht]
	\centering
	\vspace{\toppaddingoffigure}
	\includegraphics[width=1.0\textwidth]{structures/full_interpreter_struct.pdf}
	\caption{Обобщенная структура интерпретатора}
	\label{f:full_interpreter_struct}
\end{figure}

Для решения поставленной задачи, в первую очередь, необходимо в соответствии с указанными требованиями
разработать грамматику предметно-ориентированного языка и спроектировать обобщенную структуру программы интерпретатора.

\subsection{Разработка грамматики языка}

Описание языка программирования основывается на теории формальных языков.
В данном разделе проводится разработка формальной грамматики языка.


\subsubsection{Способы задания языков}

Для задания языка можно воспользоваться следующими методами:

\begin{enumerate}
    \item перечислить все цепочки языка;
    \item указать способ порождения цепочек;
    \item определить метод распознавания допустимых цепочек.
\end{enumerate}

Перечисление всех цепочек языка возможно в исключительных случаях, например,
когда для управления некоторой системой достаточно двух-трех команд.

Механизм порождения цепочек предполагает использование формальной порождающей грамматики.

Формальная порождающая грамматикиа -- это математиечская система, описывающая правила построения цепочек некоторого (формального) языка.

Распознавание допустимых цепочек осуществляется с помощью некоторого логического устройства -- распознавателя.
На вход распознавателя подается цепочка, а на выходе образуется логическое значение <<истина>> в случае принадлежности цепочки языку
и <<ложь>>, если цепочка языку не принадлежит.
Распознаватели строятся на основе теорий конечных автоматов и автоматов с магазинной памятью.

Методы порождения и распознавания тесно связаны.
Механизм порождения обычно используется при описании языка, а распозанватель при его реализации, т.е. в трансляторе.

Описать синтаксис языков программирования можно несколькими способами, например, такими как формы Бэкуса-Наура, диаграммы Вирта и другими.
Это методы задают правила вывода, определяющие возможные конструкции цепочек языка.
В данном проекте для описания грамматики языка используется расширенная форма Бэкуса-Наура (РБНФ).



\subsubsection{Применение расширенной формы Бэкуса-Наура для описания формальной грамматики языка}

Расширенная форма Бэкуса-Наура – формальная система определения синтаксиса,
в которой одни синтаксические категории последовательно определяются через другие.
Используется для описания контекстно-свободных грамматик.

Формальная грамматика задаётся четвёркой вида:

\(G = (V_T, V_N, P, S)\),

где \(V_T\) -- множество терминальных символов грамматики – конечные
элементы языка, не разбирающиеся на более мелкие составляющие в рамках
синтаксического анализа, например ключевые слова, цифры, буквы
латинского алфавита.

\(V_N\) -- конечное множество нетерминальных символов – элементов грамматики, имеющих собственные имена и структуру.
Каждый нетерминальный символ состоит из одного или более терминальных и/или нетерминальных символов.

\(P\) -- множество правил вывода грамматики.

\(S\) -- начальный символ грамматики, \(S \in V_N\).

РБНФ является одним из видов формальных грамматик.
РБНФ состоит из множества правил вывода, каждое из которых определяет синтаксис некоторой конструкции языка.

Некоторые основные конструкции РБНФ:

\begin{itemize}
    \item A, B -- конкатенация элементов;
    \item A | B -- выбор (A или B);
    \item {[A]} -- элемент в квадратных скобках может отсутствовать (аналог - <<?>>);
    \item \{A\} -- повторение элемента 0 или более раз (аналог - <<*>>);
    \item (A B) -- группировка элементов;
    \item (* … *) – комментарий;
    \item <<;>> – отмечает окончание правила (аналог - <<.>>).
\end{itemize}

Кроме того, в качестве синтаксического сахара могут использоваться следующие символы:

\begin{itemize}
    \item <<*>> - предыдущий элемент может встречаться 0 или более раз;
    \item <<?>> - предыдущий элемент является необязательным (присутствует 0 или 1 раз);
    \item <<+>> - предыдущий элемент встречается 1 или более раз.
\end{itemize}

В соответствии с данными правилами описание синтаксиса предметно-ориентированного языка будет выглядеть следующим образом:

Program = Statement+ \\

Statement = AssignStmt | FunctionDecl | ExpressionStmt | ReturnStmt | BlockStmt | IfStmt .

ExpressionStmt = Expression .

Identifier = (letter | "{}\_"{}) { letter | "{}\_"{} | digit } . \\

Expression = UnaryExpr | Expression binary\_op Expression . 

UnaryExpr = PrimaryExpr | unary\_op UnaryExpr . \\

PrimaryExpr = Operand | PrimaryExpr Index | CallExpr .

Index = "{}["{} Expression "{}]"{} . \\

AssignStmt = Identifier assign\_op Expression .

ReturnStmt = "{}return"{} [Expression] .

BlockStmt = "{}\{"{} StatementList "{}\}" .

StatementList = \{ Statement "{};"{} \} . \\

IfStmt = "{}if("{} [Expression] "{})"{} BlockStmt ["{}else"{} BlockStmt] . \\

FunctionDecl = "{}fn("{} [ParameterList] "{})"{} BlockStmt .

ParameterList = Identifier \{ "{},"{} Identifier \} .

Arguments = "{}("{} [ ExpressionList ] "{})"{} . \\

CallExpr = Identifier Arguments .

ExpressionList = Expression \{ "{},"{} Expression \} . \\

Array = "{}["{} [ ExpressionList ] "{}]"{} . \\

Key = stringLiteral | intLiteral | Identifier | Expression .

KeyedElement  = [ Key "{}:"{} Expression ] .

Map = "{}\{"{} KeyedElement \{ "{},"{} KeyedElement \} "{}\}"{} . \\

Operand = Literal | "{}("{} Expression "{})"{} .

Literal = intLiteral | stringLiteral | Array | Map .

intLiteral = digit \{ digit \} .

stringLiteral = << "{} >> \{ ascii\_char \} << "{} >> . \\

binary\_op = "{}||"{} | "{}\&\&"{} | rel\_op | add\_op | mul\_op .

rel\_op = "{}=="{} | "{}!="{} | \"{}<\"{} | "{}<="{} | "{}>"{} | "{}>="{} .

add\_op = "{}+"{} | "{}-"{} .

mul\_op = "{}*"{} | "{}/"{} | "{}\%"{} .

assign\_op = "{}="{} . \\

unary\_op = "{}-"{} | "{}!"{} . \\

digit = "{}0"{} ... "{}9"{} .

letter = "{}A"{} ... "{}Z"{} | "{}a"{} ... "{}z"{} .

ascii\_char = (* ascii character *) . \\

Начальное состояние, с которого начинается разбор -- Program.

В данном разделе была разработана и описана с помощью расширенной формы Бэкуса-Наура формальная грамматика предметно-ориентированного языка.
\subsection{Разработка лексического анализатора}

% В данном разделе необходимо выполнить разработку алгоритмов функционирования лексического анализатора.

Лексический анализ – процесс разбора входной последовательности символов на распознанные группы – лексемы.

Лексемой является структурная (минимальная значимая) единица языка, состоящая из элементарных символов языка и не содержащая в своём составе других структурных единиц языка.

В ходе выполнения лексического анализатора каждая лексема идентифицируется и преобразуется в токен.

Токен – экземпляр лексемы, представляющий собой пару «тип лексемы» и «значение».
«Тип» указывает на принадлежность лексемы к определенной категории, например, идентификатор, число и т.д., а «значение» содержит конкретные данные, соответствующие этой лексеме.

Категории токенов, которые используются в разрабатываемом предметно-ориентированном языке:
\begin{itemize}
    \item идентификаторы;
    \item числа;
    \item строки;
    \item разделители;
    \item операторы (арифметические, сравнения и т.д);
    \item скобки;
    \item специальные (конец входной последовательности и т.п)
    \item ключевые слова.
\end{itemize}

Полный список токенов с указанием категории и примерами лексем приведен в таблице~\ref{t:tokens}.

Процесс лексического анализа является первым шагов в трансляции исходного кода программы и формирует основу для следующих этапов,
таких как синтаксический анализ и построение абстрактного синтаксического дерева.

\clearpage

\begin{table}[h!]
    \Large
    \caption{Токены с примерами}
    \label{t:tokens}
    \centering
    \begin{tabularx}{\textwidth}{|X|c|c|}
        \hline
        Токен         & Категория      & Пример лексемы  \\
        \hline
        IDENT         & Идентификатор  & qwe             \\
        \hline
        INT           & Число          & 123             \\
        \hline
        STRING        & Строка         & «привет, hello» \\
        \hline
        ASSIGN        & Оператор       & =               \\
        \hline
        PLUS          & Оператор       & +               \\
        \hline
        MINUS         & Оператор       & -               \\
        \hline
        STAR          & Оператор       & *               \\
        \hline
        SLASH         & Оператор       & /               \\
        \hline
        EXCLAMINATION & Оператор       & !               \\
        \hline
        PERCENT       & Оператор       & \%              \\
        \hline
        EQ            & Оператор       & ==              \\
        \hline
        NEQ           & Оператор       & !=              \\
        \hline
        LEQ           & Оператор       & <=              \\
        \hline
        GEQ           & Оператор       & =>              \\
        \hline
        LT            & Оператор       & <               \\
        \hline
        GT            & Оператор       & >               \\
        \hline
        LAND          & Оператор       & \&\&            \\
        \hline
        LOR           & Оператор       & ||              \\
        \hline
        COMMA         & Разделитель    & ,               \\
        \hline
        SEMICOLON     & Разделитель    & ;               \\
        \hline
        LPAR          & Скобка         & (               \\
        \hline
        RPAR          & Скобка         & )               \\
        \hline
        LBRACE        & Скобка         & \{              \\
        \hline
        RBRACE        & Скобка         & \}              \\
        \hline
        LBRACKET      & Скобка         & [               \\
        \hline
        RBRACKET      & Скобка         & ]               \\
        \hline
        IF            & Ключевое слово & if              \\
        \hline
        ELSE          & Ключевое слово & else            \\
        \hline
        TRUE          & Ключевое слово & true            \\
        \hline
        FALSE         & Ключевое слово & false           \\
        \hline
        FUNC          & Ключевое слово & fn              \\
        \hline
        RETURN        & Ключевое слово & return          \\
        \hline
        ILLEGAL       & Специальный    & @               \\
        \hline
        EOF           & Специальный    & Конец файла     \\
        \hline
    \end{tabularx}
    \vspace{\bottompaddingoftable}
\end{table}

Схема взаимодействия лексического и синтаксического анализаторов показана на рисунке~\ref{f:la_sa_struct}.

\begin{figure}[ht]
	\centering
	\vspace{\toppaddingoffigure}
	\includegraphics[width=0.9\textwidth]{structures/lexical_analyzer/la_sa_struct.pdf}
	\caption{Схема взаимодействия лексического и синтаксического анализаторов}
	\label{f:la_sa_struct}
\end{figure}

При запросе нового токена лексический анализатор считывает входной поток символов до точной идентификации следующего токена.

Процесс распознавания токенов из входного потока символов языка можно показать с помощью диаграмм переходов состояний.

На рисунке~\ref{f:dps_eq_assign} показана диаграмма для определения токенов «=» и «==».

\begin{figure}[ht]
	\centering
	\vspace{\toppaddingoffigure}
	\includegraphics[width=0.9\textwidth]{structures/lexical_analyzer/dps_eq_assign.pdf}
	\caption{Диаграмма переходов для определения «=» и «==»}
	\label{f:dps_eq_assign}
\end{figure}

Работа начинается с состояния 0, в котором считывается следующий символ из входного потока.
Если полученный символ «=», то по дуге, помеченной «=» выполняется переход в состояние 1.
В состоянии 1 выполняется считывание следующего символа.
Если этот символ «=», выполняется переход в состояние 2 – заключительное состояние,
в котором найден токен «EQ», в том случае, если был получен символ отличный от «=»,
происходит переход по дуге «other» в состояние 3 с токеном «ASSIGN».

Диаграмма для распознавания целого числа представлена на рисунке~\ref{f:dps_int}.

\begin{figure}[ht]
	\centering
	\vspace{\toppaddingoffigure}
	\includegraphics[width=0.9\textwidth]{structures/lexical_analyzer/dps_int.pdf}
	\caption{Диаграмма переходов для определения целого числа}
	\label{f:dps_int}
\end{figure}

При получении в начальном состоянии цифры, выполняется переход в состояние 39, в котором автомат находится до тех пор,
пока не получит на вход символ, отличный от цифры, при получении такого символа выполняется переход в конечное состояние 40.
По мере определения очередной цифры, она заносится в буфер.
В состоянии 40 возвращается токен INT и значение числа из буфера.

Считывание ключевых слов и идентификаторов показано с помощью диаграммы передов на рисунке~\ref{f:dps_ident_kw}.

Из начального состояния происходит переход в состояние 35, если была получена буква.
По аналогии с состоянием 39 выполняется циклическое считывание букв с занесением в буфер.
Если была получена не буква, выполняется переход в состояние 36, в котором проверяется принадлежность считанной строки к списку ключевых слов.
В случае, если считанная строка является ключевым словом, автомат переходит в завершающее состояние 37 в котором указывается тип токена для полученного ключевого слова и значение.
Если в состоянии 36 проверка показала, что строка на является ключевым словом, то выполняется переход в состояние 38 – определен идентификатор.

\begin{figure}[ht]
	\centering
	\vspace{\toppaddingoffigure}
	\includegraphics[width=0.9\textwidth]{structures/lexical_analyzer/dps_ident_kw.pdf}
	\caption{Диаграмма переходов для определения идентификаторов и ключевых слов}
	\label{f:dps_ident_kw}
\end{figure}

Полная диаграмма переходов состояний представлена на рисунке~\ref{f:full_dps}

Процесс распознавания токена начинается с начального состояния 0.
В зависимости от полученного символа выполняется переход в конкретное состояние.
Однако, если в начальном состоянии был получен символ, для которого нет дуги,
по которой он бы мог перейти в определенное для него состояние, выполняется переход по дуге «other» в состояние 42 с определением токена ILLEGAL.
После определения очередного токена в конченом состоянии, автомат начинает работу заново с начального состояния.
Считывание входного потока символов прекращается при поступлении нулевого символа (null character) с определением токена EOF.
Вместе с токеном, лексический анализатор возвращает его позицию в входном коде.

% В данном разделе была выполнена разработка структурных решений и алгоритмов функционирования лексического анализатора предметно-ориентированного языка.

\clearpage

\begin{figure}[h!]
	\centering
	\vspace{\toppaddingoffigure}
	\includegraphics[width=1.0\textwidth]{structures/lexical_analyzer/full_dps.pdf}
	\caption{Полная диаграмма переходов состояний}
	\label{f:full_dps}
\end{figure}

\clearpage
\subsubsection{Реализация синтаксического анализатора}

Разработка синтаксического анализатора включает в себя программную реализацию парсера, способного анализировать токены, получаемые от лексического анализатора и строить абстрактное синтаксическое дерево.

Абстрактное синтаксическое дерево представляет собой структуру, отражающую синтаксическую структуру программы.
Узлы AST могут быть двух типов: statement -- инструкции и expression -- выражения. 
В соответствии с этим, их программная реализация выполнена в виде интерфейсов, представленных на рисунке~\ref{f:code_astInterfaces}.

\begin{figure}[ht]
	\centering
	\vspace{\toppaddingoffigure}
	\begin{lstlisting}[
        language=Go
    ]
type Node interface {
    TokenLiteral() string
    ToString() string
}

// All statement nodes implement
type Statement interface {
    Node
    statementNode()
}

// All expression nodes implement
type Expression interface {
    Node
    expressionNode()
} 
\end{lstlisting}
	\caption{Интерфейсы узлов AST}
	\label{f:code_astInterfaces}
\end{figure}

Узлы дерева состоят из интерфейса Node.
Однако сам по себе он не используется в AST, а необходим для расширения двух вспомогательных интерфейсов Statement и Expression, которые определяют узлы двух типов: инструкции и выражения соответственно.

Пример кода структуры AssignStatement, реализующей интерфейс Statement на рисунке~\ref{f:code_IStatementExample}.

Пример кода структуры IntegerLineral, реализующей интерфейс Expression представлен на рисунке~\ref{f:code_IExpressionExample}.

\begin{figure}[!htb]
	\centering
	\vspace{\toppaddingoffigure}
	\begin{lstlisting}[
        language=Go,
        xleftmargin=.08\textwidth,
        xrightmargin=.08\textwidth
    ]
type AssignStatement struct {
    Name  *Ident
    Value Expression
}

func (as *AssignStatement) statementNode()       {}
func (as *AssignStatement) TokenLiteral() string { return "" }
func (as *AssignStatement) ToString() string {
    var out bytes.Buffer

    out.WriteString(as.Name.TokenLiteral())
    out.WriteString(" = ")

    if as.Value != nil {
        out.WriteString(as.Value.ToString())
    }

    out.WriteString(";")

    return out.String()
}
\end{lstlisting}
	\caption{Пример реализации интерфейса Statement}
	\label{f:code_IStatementExample}
\end{figure}

\begin{figure}[!htb]
	\centering
	\vspace{\toppaddingoffigure}
	\begin{lstlisting}[
        language=Go,
        xleftmargin=.08\textwidth,
        xrightmargin=.08\textwidth
    ]
type IntegerLiteral struct {
    Token token.Token // 5 6
    Value int64
}

func (il *IntegerLiteral) expressionNode()      {}
func (il *IntegerLiteral) TokenLiteral() string { return il.Token.Literal }
func (il *IntegerLiteral) ToString() string     { return il.Token.Literal }
\end{lstlisting}
	\caption{Пример реализации интерфейса Expression}
	\label{f:code_IExpressionExample}
\end{figure}

Рассмотрим пример работы синтаксического анализатора.

Входная строка: $5 + 1 * 2 / (4 + 9)$.

Результат работы синтаксического анализатора в виде строки с исходным кодом программы,
в котором с помощью скобок обозначены приоритеты операторов представлена на рисунке~\ref{f:parserCodeResult}.

\begin{figure}[ht]
	\centering
	\vspace{\toppaddingoffigure}
	\includegraphics[width=0.7\textwidth]{parser/parserCodeResult.png}
	\caption{Результат работы синтаксического анализатора в виде строки}
	\label{f:parserCodeResult}
\end{figure}

Графическое представление AST для указанных входных данных представлено на рисунке~\ref{f:ast}.
Дерево, сформированное в результате работы программы для указанных входных данных представлен на рисунке~\ref{f:astRawCmd}.

\begin{figure}[ht]
	\centering
	\vspace{\toppaddingoffigure}
	\includegraphics[width=0.7\textwidth]{parser/ast.pdf}
	\caption{AST для указанной входной строки}
	\label{f:ast}
\end{figure}

\clearpage

\begin{figure}[!htb]
	\centering
	\includegraphics[width=0.48\textwidth]{parser/astRawCmd.pdf}
	\caption{Результат работы парсера в виде AST}
	\label{f:astRawCmd}
\end{figure}

\clearpage
\subsubsection{Реализация семантического анализатора}

Прежде, чем переходить непосредственно к реализации семантического анализатора, необходимо реализовать объектную систему.
Объектная система является основополагающей частью семантического анализатора и исполнителя.
Семантической анализатор в процессе своей работы выполняет необходимые проверки на основе значений,
представленных в виде объектов внутреннего представления.

Каждое значение выражения представляется в виде структуры, которая соответствует некоторому объекту интерфейсного типа -- рисунок~\ref{f:code_ObjectInterface}.

\begin{figure}[ht]
	\centering
	\vspace{\toppaddingoffigure}
	\begin{lstlisting}[
        language=Go
    ]
type ObjectType string

type Object interface {
    Type() ObjectType
    ToString() string
}
\end{lstlisting}
	\caption{Интерфейс объекта}
	\label{f:code_ObjectInterface}
\end{figure}

На рисунке~\ref{f:code_IObjectExample} приведен пример структуры, которая представляет данные типа «целое число».

\begin{figure}[ht]
	\centering
	\vspace{\toppaddingoffigure}
	\begin{lstlisting}[
        language=Go,
        xleftmargin=.08\textwidth,
        xrightmargin=.08\textwidth
    ]
func (i *Integer) Type() ObjectType { return INTEGER_OBJ }
func (i *Integer) ToString() string {
    return fmt.Sprintf("%d", i.Value)
}
func (i *Integer) HashKey() HashKey {
    return HashKey{Type: i.Type(), Value: uint64(i.Value)}
}
\end{lstlisting}
	\caption{Пример реализации интерфейса Object}
	\label{f:code_IObjectExample}
\end{figure}

Поле «Value» предназначено для хранения значения числа.
Метод «Type()» возвращает информацию о принадлежности структуры типу «integer». 
Метод «ToString()» формирует хранимое значение в виде строки.
Используется для читаемого представления значения объекта в процессе отладки. 

Подобные структуры, реализующие интерфейс «Object» представлены для всех примитивных и составных типов данных, используемых в языке:
boolean, string, integer, array, HashMap.
Кроме этого, реализованы еще несколько вспомогательных структур:

\begin{itemize}
    \item «Null» для поддержки соответствующих значений;
    \item «Error», содержащая информацию об ошибке, возникшей на этапе семантического анализа;
    \item «Return» для представления возвращаемых значений;
    \item «Function» - специальная структура, используемая при обработке вызова функции;
    \item «Builtin» - структура, представляющая встроенные функции.
\end{itemize}

Данные объекты формируют объектную систему внутреннего представления значений программы.

Семантический анализ выполняется во время рекурсивного прохода по узлам AST.
В ходе рекурсии при достижении примитивных значений в крайних узлах ветвей AST для каждого значения создается объект  внутреннего представления соответствующего типа.
На обратном ходу рекурсии, при необходимости выполнить проверку семантической корректности выражения она выполняется над объектами внутреннего представления, сформированными ранее.
При анализе генерируется ошибка, в случае её обнаружения, в противном случае начинается вычисление значение выражения исполнителем.

Фрагмент кода функции разбора инфиксного выражения представлен на рисунке~\ref{f:code_semantic}.

\begin{figure}[ht]
	\centering
	\vspace{\toppaddingoffigure}
	\begin{lstlisting}[
        language=Go,
        xleftmargin=.08\textwidth,
        xrightmargin=.03\textwidth
    ]
func evalInfixExpression(op string, left object.Object, right object.Object) object.Object {
    switch {
    case left.Type() != right.Type():
        return newError("type mismatch: %s %s %s", left.Type(), op, right.Type())
    case left.Type() == object.INTEGER_OBJ && right.Type() == object.INTEGER_OBJ:
        return evalIntInfixExpr(op, left, right)
    case left.Type() == object.STRING_OBJ && right.Type() == object.STRING_OBJ:
        return evalStringInfixExpr(op, left, right)
    case op == "==":
        return boolToBooleanObj(left == right)
    case op == "!=":
        return boolToBooleanObj(left != right)
    case op == "||":
        return boolToBooleanObj(left.(*object.Boolean).Value || right.(*object.Boolean).Value)
    case op == "&&":
        return boolToBooleanObj(left.(*object.Boolean).Value && right.(*object.Boolean).Value)
    default:
        return newError("unknown operator: %s %s %s", left.Type(), op, right.Type())
    }
}
\end{lstlisting}
	\caption{Пример семантического анализа инфиксного выражения}
	\label{f:code_semantic}
\end{figure}
\subsection{Реализация исполнителя}

В общем виде процесс исполнения тесно связан с этапом семантического анализа.
Выполняется рекурсивный обход абстрактного синтаксического дерева.
Первым шагом каждое выражение проходит семантическую проверку.
После успешного завершения семантического анализа выражения из AST передаются на этап их вычисления.
Разнотипные выражения обрабатываются по-разному, однако результат вычисления всегда представляет собой некоторый тип данных, представленный в виде объекта -- внутреннего представления.

Окружение для хранения информации о переменных в программном коде реализовано в виде структуры,
содержащей поля с хэш-картой с самими переменными и их значениями, а также ссылка на эту же структуру для организации области видимости при вызове функций.
На рисунке~\ref{f:code_envStruct} приведен пример программной реализации данной структуры.

\begin{figure}[ht]
	\centering
	\vspace{\toppaddingoffigure}
	\begin{lstlisting}[
        language=Go
    ]
type Env struct {
    store map[string]Object
    outer *Env
}
\end{lstlisting}
	\caption{Реализация окружения}
	\label{f:code_envStruct}
\end{figure}

За обработку узлов AST отвечает единственная функция, которая определяет тип узла и передает управление соответствующей функции,
которая после прохождения семантической проверки вычисляет значение для узла данного типа.
В конечном итоге формируется внутреннее представление в виде объектов с примитивными типами данных, либо объекты представляющие составные типы, состоящие из примитивных, так как массивы и хеш-карты.
Фрагмент кода основной функции получения и обработки узлов AST приведен на рисунке~\ref{f:code_evalFragment}.

При обнаружении в коде определения переменной, ее необходимо сохранить в памяти, чтобы в дальнейшем иметь к ней доступ.
Для этого используется окружение.
Переменные в окружении хранятся в виде хэш-карты, ключи которой представляют идентификатор переменной, а значения – внутреннее представление значений переменной, то есть объекты.

Код функций записи переменной в окружение и получения из него приведен на рисунке~\ref{f:code_getsetEnv}.

\clearpage

\begin{figure}[!htb]
	\centering
	\begin{lstlisting}[
        language=Go,
        xleftmargin=.08\textwidth,
        xrightmargin=.08\textwidth
    ]
func Eval(n ast.Node, env *object.Env) object.Object {
    switch node := n.(type) {
    case *ast.Program:
        return evalProgram(node, env)
    case *ast.BlockStatement:
        return evalBlockStatement(node, env)
    case *ast.ExpressionStatement:
        return Eval(node.Expression, env)
    case *ast.ReturnStatement:
        val := Eval(node.Value, env)
        if isError(val) {
            return val
        }

        return &object.Return{Value: val}
    case *ast.AssignStatement:
        val := Eval(node.Value, env)
        if isError(val) {
            return val
        }

        env.Set(node.Name.Value, val)
    case *ast.IntegerLiteral:
        return &object.Integer{Value: node.Value}
    case *ast.Boolean:
        if node.Value {
            return TRUE
        }
        return FALSE
\end{lstlisting}
	\caption{Фрагмент кода исполнителя}
	\label{f:code_evalFragment}
\end{figure}

\begin{figure}[!htb]
	\centering
	\begin{lstlisting}[
        language=Go,
        xleftmargin=.08\textwidth,
        xrightmargin=.08\textwidth
    ]
func (e *Env) Get(key string) (Object, bool) {
    obj, ok := e.store[key]
    if !ok && e.outer != nil {
        obj, ok = e.outer.Get(key)
    }
    return obj, ok
}
func (e *Env) Set(key string, val Object) Object {
    e.store[key] = val
    return val
}    
\end{lstlisting}
	\caption{Код функций записи и получения значений окружения}
	\label{f:code_getsetEnv}
\end{figure}

\clearpage


Использование единого хранилища значений переменных для всей области видимости программы вносит некоторые ограничения.
Например в аргументах функции могут быть определены параметры, имена которых совпадают с объявленным ранее переменным.
Это некорректное поведение, так как первое объявленное значение будет перезаписано другим, переданным в функцию.
Пример кода такой ситуации приведен на рисунке~\ref{f:code_rewriteVarError}.

\begin{figure}[!htb]
	\centering
    \vspace{\toppaddingoffigure}
	\begin{lstlisting}[
        language=Go,
        xleftmargin=.08\textwidth,
        xrightmargin=.08\textwidth
    ]
x = 10;
f = func(x) {
    return x * 10;
}
f(5);
x; //5
\end{lstlisting}
	\caption{Пример кода некорректного поведения}
	\label{f:code_rewriteVarError}
\end{figure}

В данном коде, имя параметра функции совпадает с именем переменной -- x.
Переменная <<x>> объявлена со значением, равным 10.
Данное значение перезапишется значением, переданным в функцию при ее вызове, в данном случае значением, равным 5.
Таким образом, после вызова функции значение переменной <<x>> будет неявно изменено и станет равным 5.
Данная логика работы является ошибочной.

Решение лежит в выделении внутреннего окружения функции при её вызове.
Именно для такого случая в ранее рассмотренной структуре Env содержится ссылка на другой экземпляр структуры такого же типа.
Так, при вызове функции, необходимо создать новый экземпляр окружения, записать в него переданные аргументы и установить ссылку на внешнее окружение,
то из которого была вызвана функция. Такой подход позволит корректно выполнять вложенные функции и рекурсивные вызовы.

Для строк и массивов реализованы несколько встроенных функций:

\begin{itemize}
    \item len -- определение длины строки или массива;
    \item push -- добавление элемента в конец массива;
    \item first -- получение первого элемента массива;
    \item last -- получение последнего элемента массива.
\end{itemize}
% Код реализации встроенных функций приведен в приложении Г.

Хэш-карты реализованы на базе хэш-карт языка Go.
В качестве ключа могут быть следующие типы данных: строка, булево значение, число.
Так как данные типы представлены в виде внутренних объектов, нельзя брать тип Object в качестве ключа.
При занесении значения в хэш-карту и попытке последующего его получения ключи будут представлять разные экземпляры несмотря на одинаковое значение.
На рисунке~\ref{f:code_hashmapGetValueExample} приведен наглядный пример.

\begin{figure}[!htb]
	\centering
    \vspace{\toppaddingoffigure}
	\begin{lstlisting}[
        language=Go
    ]
X = { "name": "Bob" }
X["name"]     
\end{lstlisting}
	\caption{Пример получения значения хэш-карты}
	\label{f:code_hashmapGetValueExample}
\end{figure}

Строковой ключ «name» при попытке получить значение из карты не будет возвращать значение "Bob",
так как при вычислении выражения будет создан новый экземпляр объекта, представляющего строку.
Для решения этой проблемы можно использовать в качестве ключа строку, содержащую хэш от значения объекта.

Фрагмент кода объектной системы с реализацией хэш-карт представлен на рисунке~\ref{f:code_hashmapObject}.

Фрагмент кода функции вычисления целочисленного инфиксного выражения приведен на рисунке~\ref{f:code_evalIntInfixExpr}.

\clearpage

\begin{figure}[!htb]
	\centering
	\begin{lstlisting}[
        language=Go,
        xleftmargin=.08\textwidth,
        xrightmargin=.08\textwidth
    ]
type HashKey struct {
    Type  ObjectType
    Value uint64
}

type Hashable interface {
    HashKey() HashKey
}

type HashPair struct {
    Key   Object
    Value Object
}

type HashMap struct {
    Pairs map[HashKey]HashPair
}

func (h *HashMap) Type() ObjectType { return HASH_MAP_OBJ }
func (h *HashMap) ToString() string {
    var out bytes.Buffer

    pairs := []string{}
    for _, el := range h.Pairs {
        pairs = append(pairs, fmt.Sprintf("%s: %s", el.Key.ToString(), el.Value.ToString()))
    }

    out.WriteString("[")
    out.WriteString(strings.Join(pairs, ", "))
    out.WriteString("]")

    return out.String()
} 
\end{lstlisting}
	\caption{Фрагмент кода реализации хэш-карт}
	\label{f:code_hashmapObject}
\end{figure}

\clearpage

\begin{figure}[!htb]
	\centering
	\begin{lstlisting}[
        language=Go,
        xleftmargin=.08\textwidth,
        xrightmargin=.08\textwidth
    ]
func evalIntInfixExpr(op string, left object.Object, right object.Object) object.Object {
    lVal := left.(*object.Integer).Value
    rVal := right.(*object.Integer).Value
    switch op {
    case "+":
        return &object.Integer{Value: lVal + rVal}
    case "-":
        return &object.Integer{Value: lVal - rVal}
    case "*":
        return &object.Integer{Value: lVal * rVal}
    case "/":
        return &object.Integer{Value: lVal / rVal}
    case "%":
        return &object.Integer{Value: lVal % rVal}
    case "==":
        return boolToBooleanObj(lVal == rVal)
    case "!=":
        return boolToBooleanObj(lVal != rVal)
    case "<":
        return boolToBooleanObj(lVal < rVal)
    case ">":
        return boolToBooleanObj(lVal > rVal)
    case "<=":
        return boolToBooleanObj(lVal <= rVal)
    case ">=":
        return boolToBooleanObj(lVal >= rVal)
    default:
        return newError("unknown operator: %s %s %s", left.Type(), op, right.Type())
    }
}
\end{lstlisting}
	\caption{Фрагмент кода функции вычисления целочисленного инфиксного выражения}
	\label{f:code_evalIntInfixExpr}
\end{figure}

% Фрагменты кода исполнителя приведены в приложении В.

Пример успешного выполнения программы для указанных входных данных представлен на рисунке~\ref{f:evalSuccessExample}.

Входная строка: $5 + 1 * 20 / (5 + 5)$.

\clearpage

\begin{figure}[!htb]
	\centering
	\includegraphics[width=0.4\textwidth]{evaluator/evalSuccessExample.png}
	\caption{Пример успешного выполнения программы}
	\label{f:evalSuccessExample}
\end{figure}

Пример завершения работы программы с семантической ошибкой для указанных ниже входных данных показан на рисунке~\ref{f:evalErrorExample}.

Входная строка представлена на рисунке~\ref{f:code_evalErrorExample}.

\begin{figure}[!htb]
	\centering
    \vspace{\toppaddingoffigure}
	\begin{lstlisting}[
        language=Go,
        xleftmargin=.08\textwidth,
        xrightmargin=.08\textwidth
    ]
x = 5 + 1 * 20 / (5 + 5)
if (x > true) {
    return 1
}    
\end{lstlisting}
	\caption{Пример кода, содержащего ошибку}
	\label{f:code_evalErrorExample}
\end{figure}

\begin{figure}[!htb]
	\centering
	\includegraphics[width=0.8\textwidth]{evaluator/evalErrorExample.png}
	\caption{Пример завершения работы программы с ошибкой}
	\label{f:evalErrorExample}
\end{figure}

\pagebreak

\section*{Вывод}

В данном разделе были разработаны архитектурно-структурные решения по поставленной задаче.
Рассмотрена модульная структура серверной части конструктора,
определена связь между модулем компонентов и модулем интерпретации предметно-ориентированного языка.

Рассмотрена обобщенная структура интерпретатора и этапы его работы.

Выполнен обзор способов задания языков, разработана и описана с помощью расширенной формы Бэкуса-Наура формальная грамматика предметно-ориентированного языка.
Выполнена разработка структурных решений и алгоритмов функционирования каждого из этапов интерпретатора, а именно:
лексического анализатора, синтаксического анализатора, семантического анализатора и исполнителя.
\subsection{Расширенное техническое задание}

В данном разделе представлено техническое задание на разработку предметно-ориентированного языка для визуального конструктора.

\subsubsection{Основание для разработки}

Программа разрабатывается на основе учебного плана кафедры
"Электронные вычислительные машины" по направлению 09.03.01

\subsubsection{Цель и задача разработки}

Целью разработки является создание языка для расширения
функциональных возможностей визуального конструктора.

Задача разработки -- проектирование и разработки программного продукта по расширению функциональных возможностей визуального конструктора
на базе предметно-ориентированного языка. 

\subsubsection{Краткая характеристика области применения}

Программный продукт предоставляет пользователям возможность более гибкого создания приложений в среде визуального программирования.

\subsubsection{Назначение разработки}

Функциональным назначением является интерпретация кода предметно-ориентированного языка, написанного пользователем визуального конструктора.

Программа-интерпретатор является модулем визуального конструктора и эксплуатируется как его составная часть.
Особые требования к конечному пользователю не предъявляются.

\subsubsection{Требования к программного продукту}

Предметно-оринетированный язык конструктора должен обладать обозначенными ниже характеристиками.

Ключевые конструкции языка: переменные, ветвления, функции.

Поддерживаемые простые типы данных: строки, числа, булевы значения, <<Null>>.

Поддерживаемые составные типы данных: массив, хэш-карта.

Набор возможных операций языка:
\begin{itemize}
	\item арифметические операции: сложение, вычитание, умножение, деление, операция нахождение остатка от деления;
	\item логические операции: конъюнкция, дизъюнкция, отрицание;
	\item операции сравнения: равно, не равно, больше, меньше, больше или равно, меньше или равно;
	\item управляющие операции;
	\item вызов функций.
\end{itemize}

Интерпретатор предметно-ориентированного языка должен поддерживать все вышеперечисленные конструкции.

В задачи интерпретатора входит выполнение следующих этапов:
\begin{itemize}
	\item лексический анализ;
	\item синтаксический анализ;
	\item семантический анализ;
	\item исполнение операций;
\end{itemize}

Исходными данными является код программы на предметно-ориентированном языке.

Выходными данными является значение, полученное в результате успешного исполнения кода.

В случае возникновения ошибки на каком-либо из этапов интерпретации программа должна возвращать человекочитаемую ошибку.

\subsubsection{Требования к надежности}

Программа должна функционировать в соответствии с заданными требованиями при отсутствии сбоев технических средств.

\section*{Вывод}

В данном разделе был проведен анализ предметной области, в результате чего было определено,
что визуальные конструкторы значительно упрощают процесс создания и запуска приложений,
однако, они имеет функциональные ограничения, которые препятствуют разработке продукта со сложной логикой работы.
Предметно-ориентированный язык позволяет расширить возможности визуального конструктора и создавать уникальные приложения,
в соответствии с пользовательскими потребностями.

Также было рассмотрено расширенное техническое задание,
в котором были определены цели и задачи разработки и основные требования к разрабатываемому программному продукту.

Основываясь на этом можно приступить к разработке предметно-ориентированного языка для визуального конструктора,
так как рассмотренная проблема является актуальной.
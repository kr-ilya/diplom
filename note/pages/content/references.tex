\newpage

\docappendix[справочное]{Библиографический список}

\begin{references}
	\urlstyle{same}
	
	\item\label{ref:nolowcode} Low-code и No-code: как программировать без кода [Электронный ресурс]. --
	Режим доступа: \url{https://blog.sf.education/low-code-i-no-code/}
	
	\item\label{ref:dsl} Предметно-ориентированный язык — Википедия [Электронный ресурс]. --
	Режим доступа: \url{https://ru.wikipedia.org/wiki/Предметно-ориентированный_язык}
	
	\item\label{ref:dsl_classification} Классификация предметно-ориентированных языков и языковых инструментариев [Электронный ресурс] --
	Режим доступа: \url{https://www.hse.ru/data/2013/01/21/1305680244/Сухов-Классификация.pdf}.

	\item\label{ref:grammar} Вл. Пономарев. Конспективное изложение теории языков программирования и методов трансляции.
	Учебно-методическое пособие. В 4-х книгах. Книга 1.
	Формальные языки и грамматики [Текст]. -- Озерск: ОТИ НИЯУ МИФИ, 2019. -- 42 с.: ил.

	\item\label{ref:rbnf} Расширенная форма Бэкуса — Наура — Википедия [Электронный ресурс]. --
	Режим доступа: \url{https://ru.wikipedia.org/wiki/Расширенная_форма_Бэкуса_—_Наура}.

	\item\label{ref:lexlem} Молчанов А. Ю. Системное программное обеспечение.
	Учебник для вузов. 3-е изд [Текст]. -- СПб.: Питер, 2010. -- 400 с.: ил.

	\item\label{ref:dragon} Ахо А. В. Компиляторы: принципы, технологии и инструментарий [Текст] /
	А. В. Ахо, М. С. Лам, Р. Сети, Д. Д. Ульман. -- 2-е изд. -- М: Вильямс, 2018. -- 1184 с.: ил.

	\item\label{ref:ganicheva} Ганичева О.Г. Теория языков программирования и методы трансляции.
	Учебное пособие [Текст]. -- Череповец: ЧГУ, 2011. -- 185 с.: ил.

	\item\label{ref:golang} Go — Википедия [Электронный ресурс]. --
	Режим доступа: \url{https://ru.wikipedia.org/wiki/Go}.

	\item\label{ref:testing} Различные виды тестирования программного обеспечения | AppMaster [Электронный ресурс]. --
	Режим доступа: \url{https://appmaster.io/ru/blog/vidy-testirovaniia-programmnogo-obespecheniia}.

	\item\label{ref:go-testing} Тестирование в Go | AppMaster [Электронный ресурс]. --
	Режим доступа: \url{https://appmaster.io/ru/blog/testirovanie-v-go}.

	\label{ref:total}
\end{references}

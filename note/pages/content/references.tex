\newpage

\docappendix{Библиографический список}

\begin{references}
	\urlstyle{same}
	
	\item\label{ref:nolowcode} Low-code и No-code: как программировать без кода [Электронный ресурс]. --
	Режим доступа: \url{https://blog.sf.education/low-code-i-no-code/}
	
	\item\label{ref:dsl} Предметно-ориентированный язык — Википедия [Электронный ресурс]. --
	Режим доступа: \url{https://ru.wikipedia.org/wiki/Предметно-ориентированный_язык}
	
	\item\label{ref:dsl_classification} Классификация предметно-ориентированных языков и языковых инструментариев [Электронный ресурс] --
	Режим доступа: \url{https://www.hse.ru/data/2013/01/21/1305680244/Сухов-Классификация.pdf}.

	\item\label{ref:grammar} Вл. Пономарев. Конспективное изложение теории языков программирования и методов трансляции.
	[Текст]. -- Учебно-методическое пособие. В 4-х книгах. Книга 1.
	Формальные языки и грамматики. -- Озерск: ОТИ НИЯУ МИФИ, 2019. -- 42 с.: ил.

	\item\label{ref:rbnf} Расширенная форма Бэкуса — Наура — Википедия [Электронный ресурс]. --
	Режим доступа: \url{https://ru.wikipedia.org/wiki/Расширенная_форма_Бэкуса_—_Наура}.

	\item\label{ref:lexlem} Молчанов А. Ю. Системное программное обеспечение
	[Текст]. -- Учебник для вузов. 3-е изд. -- СПб.: Питер, 2010. -- 400 с.: ил.

	\item\label{ref:dragon} Ахо А. В. Компиляторы: принципы, технологии и инструментарий / А. В. Ахо, М. С. Лам, Р. Сети, Д. Д. Ульман
	[Текст]. -- 2-е изд. -- М: Вильямс, 2018. -- 1184 с.: ил.

	\item\label{ref:ganicheva} Ганичева О.Г. Теория языков программирования и методы трансляции
	[Текст]. -- Учебное пособие. -- Череповец: ЧГУ, 2011. -- 185 с.: ил.

	\item\label{ref:golang} Go — Википедия [Электронный ресурс]. --
	Режим доступа: \url{https://ru.wikipedia.org/wiki/Go}.

	\item\label{ref:testing} Различные виды тестирования программного обеспечения | AppMaster [Электронный ресурс]. --
	Режим доступа: \url{https://appmaster.io/ru/blog/vidy-testirovaniia-programmnogo-obespecheniia}.

	\item\label{ref:go-testing} Тестирование в Go | AppMaster [Электронный ресурс]. --
	Режим доступа: \url{https://appmaster.io/ru/blog/testirovanie-v-go}.

	% Караваева О. В. Операционные системы. Управление памятью и
	% процессами [Текст]: Учебно-методическое пособие / О. В. Караваева. –
	% Киров: ФГБОУ ВО «ВятГУ», 2016. – 39 с

	% \item\label{ref:123} Бахвалов Н.С., Жидков Н.П., Кобельников Г.М.
	% Численные методы [Текст] – 4-е изд. – М:. БИНОМ. Лаборатория
	% знаний, 2006. – 636 с.: ил.


	% \item Безрученко Б.П., Смирнов Д.А.
	% Статистическое моделирование по временным рядам [Электронный ресурс]
	% Cарат. отд-ние Ин-та радиотехники и электроники РАН.
	% – Электрон. дан. – Саратов, 2000. – Режим доступа:
	% http://www.masters.donntu.edu.ua/2012/fknt/dorosh/library/article4.pdf,
	% свободный. - Загл. с экрана.
	% \item Перельмутер А.В. Расчетные модели сооружений и возможность их анализа [Текст] / А.В. Перельмутер, В.И.
	% Сливкер. Киев: Сталь, 2002. 600 с.
	% \item NeuroShell 2 [Электронный ресурс] – Режим доступа: http://www.neuroproject.ru/aboutproduct.php, свободный. –
	% Загл. с экрана.
	% \item\label{ref:time-series-analysis} Бокс Дж., Дженкинс Г. Анализ временных рядов.
	% Прогноз и управление [Текст]. - М.: Мир, 1974.

	\label{ref:total}
\end{references}

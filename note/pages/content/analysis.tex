
\section{Анализ предметной области}

В данном разделе проводится анализ предметной области,
который позволит обосновать актуальность разработки проекта,
приводятся ключевые требования и особенности конструкций,
которые должны быть реализованы в предметно-ориентированном языке визуального конструктора Telegram ботов.

Боты в Telegram являются его популярной особенностью.
С их помощью пользователи могут в интуитивно понятной форме выполнять различные действия, не выходя из мессенджера Telegram.

Боты могут обладать различным функционалом и использоваться в различных сферах, например:
\begin{itemize}
    \item боты для общения с клиентами;
    \item техническая поддержка;
    \item продажа товаров и услуг;
    \item образовательные боты;
    \item боты для знакомств и общения;
    \item развлечения;
    \item утилиты и интрументы.
\end{itemize}

Боты имеют множетство плюсов как для пользователей, так и для их владельцев, например некоторые из них:
\begin{itemize}
    \item замена мобильного приложения;
    \item удобство использования, интерактивное взаимодействие;
    \item интеграция с другими системами;
    \item снижение затрат;
    \item круглосуточный доступ.
\end{itemize}

С ростом популярности ботов на рынке стали появляться визуальные конструкторы Telegram ботов --
решения для разработки и запуска ботов ботов с минимальным написанием программного кода или совсем без него. 

\subsection{Актуальность разработки}

Визуальным конструктором называется NoCode/LowCode инструмент,
предназначенный для быстрого создания приложений без обязательного знания языков программирования общего назначения.
Иными словами, весь процесс разработки -- это взаимодействие с визуальными компонентами платформы,
с помощью которых выстраивается логика работы приложения.
За счет этого конструкторы значительно упрощают и удешевляют разработку и запуск программных продуктов.
Ведь не все обладают знаниями и навыками программирования с использованием языков общего назначения, достаточными для создания даже простых программ.
Кроме того, при наличии конструктора нет необходимости разрабатывать каждый раз отдельное приложение для выполнения типовых задач,
так как конструктор предоставляет необходимый набор инструментов для быстрого создания прототипа.

Конструкторы имеют некоторые ограничения, например, при их использовании нельзя выйти за рамки возможностей самого конструктора,
а при выходе нового функционала Telegram Bot API,
владельцам платформы визуального конструирования ботов потребуется некоторые время на реализацию поддержки новых методов.

Однако использование только визуальных инструментов накладывает некоторые функциональные ограничения.
Обычно количество предоставляемых конструктором компонентов невелико и каждый из них способен выполнять только некоторую небольшую функцию, например отправить сообщение.
Это значительно ограничивает возможности пользователя в создании уникальных ботов.
Чтобы создание Telegram бота было более гибким, в систему можно интегрировать предметно-ориентированный язык программирования,
направленный на расширение функциональных возможностей визуального конструктора.
В отличие от визуальных блоков – язык позволяет пользователям сервиса гибко описывать логику работы бота.

Предметно-ориентированный язык (domain-specific language, DSL) -- это компьютерный язык, специализированный для конкретной предметной области применения.
Противоположностью DSL являются языки общего назначения, такие как C++, Python, Go и т.д.

Примеры предметно-ориентированных языков:
\begin{itemize}
    \item язык запросов SQL -- применяется при работе с базами данных;
    \item shell-скрипты;
    \item HTML -- язык разметки пользовательского веб-интерфейса;
    \item CSS -- каскадные таблицы стилей, описывающие внешний вид веб-страницы.
\end{itemize}

Предметно-ориентированные языка можно разделить на две группы по способу представления конструкций:
\begin{enumerate}
    \item текстовые DSL -- текстовая форма, по аналогии с языками общего назначения;
    \item визуальные DSL -- формирование конструкций выполняется в графическом виде.
\end{enumerate}

Визуальные DSL получили большее распространение, поскольку графическое представление информации обладает большей наглядностью.

Также DSL делятся на два типа: внутренние и внешние.

Внутренние языки опираются на язык общего назначения, являеются его часть и дополняют его.
Синтаксис такого DSL не может нарушать синтаксис базового языка.

Внешние DSL являются самостоятельными языками, имеют свой синтаксис и семантику.
Для успешного запуска они имеют свой компилятор или интерпретатор.

DSL разрабатываются с учетом особенностей предметной области,
благодаря чему являются менее избыточными по сравнению с языками общего назначения и более понятными для специалистов данной области.
Также предметно-ориентированные языки позволяют работать на более высоком уровне абстракции,
что увеличивает эффективность решения поставленных задач и снижает необходимость в изучении универсальных языков.
DSL языки легче изучать, учитывая их ограниченную область применения.

Помимо приведённых положительных аспектов предметно-ориентированных языков, они имеют некоторые недостатки.
DSL по сравнению с языками общего назначения имеют ограниченные возможности, например, малое разнообразие алгоритмов и структур данных.
Кроме того, разработка и внедрение предметно-ориентированного языка может привести к значительным тратам временных и финансовых ресурсов.

\subsection{Расширенное техническое задание}

В данном разделе представлено техническое задание на разработку предметно-ориентированного языка для конструктора Telegram ботов.

\subsubsection{Основание для разработки}

Программа разрабатывается на основе учебного плана кафедры
<<Электронные вычислительные машины>> по направлению 09.03.01.



\subsubsection{Цель и задача разработки}

Целью разработки является создание языка для расширения
функциональных возможностей визуального конструктора.

Задача разработки -- проектирование и разработки программного продукта по расширению функциональных возможностей визуального конструктора
на базе предметно-ориентированного языка. 



\subsubsection{Краткая характеристика области применения}

Программный продукт предоставляет пользователям возможность более гибкого создания приложений в среде визуального построения Telegram ботов.



\subsubsection{Назначение разработки}

Функциональным назначением является интерпретация кода предметно-ориентированного языка, написанного пользователем визуального конструктора.

Программа-интерпретатор является модулем визуального конструктора и эксплуатируется как его составная часть.
Особые требования к конечному пользователю не предъявляются.



\subsubsection{Требования к программному продукту}

Конструктор ботов делится на клиентскую и серверную части.
Серверная часть реализует основной функционал конструктора и логику его работы,
а так же предоставляет API интерфейс для клиентской части.
Клиентская часть предоставляет собой тонкий клиент в виде пользовательский веб-интерфейса.

Предметно-оринетированный язык должен быть выполнен в виде подключаемого модуля к серверной части платформы.
Модуль должен иметь программный интерфейс для запуска интерпретации кода предметно-оринетированного языка и возврата результата.
Помимо передачи кода на интерпретацию необходимо предусмотреть возможность передачи значений внешних переменных, используемых в передаваемом коде.

На клиентской части платформы должен быть реализован визуальный компонент,
позволяющий пользователю конструктора писать и редактировать код на предметно-оринтированном языке
и указывать внешние переменные, необходимые для успешного запуска кода.

Предметно-оринетированный язык конструктора должен обладать обозначенными ниже характеристиками.

Ключевые конструкции языка: переменные, ветвления, функции.

Поддерживаемые простые типы данных: строки, числа, булевы значения, <<Null>>.

Поддерживаемые составные типы данных: массив, хэш-карта.

Набор возможных операций языка:
\begin{itemize}
	\item арифметические операции: сложение, вычитание, умножение, деление, операция нахождение остатка от деления;
	\item логические операции: конъюнкция, дизъюнкция, отрицание;
	\item операции сравнения: равно, не равно, больше, меньше, больше или равно, меньше или равно;
	\item управляющие операции;
	\item вызов функций.
\end{itemize}

Интерпретатор предметно-ориентированного языка должен поддерживать все вышеперечисленные конструкции.

В задачи интерпретатора входит выполнение следующих этапов:
\begin{itemize}
	\item лексический анализ;
	\item синтаксический анализ;
	\item семантический анализ;
	\item исполнение операций;
\end{itemize}

Исходными данными является код программы на предметно-ориентированном языке.

Выходными данными является значение, полученное в результате успешного исполнения кода.

В случае возникновения ошибки на каком-либо из этапов интерпретации программа должна возвращать человекочитаемую ошибку.



\subsubsection{Требования к надежности}

Программа должна функционировать в соответствии с заданными требованиями при отсутствии сбоев технических средств.


\section*{Вывод}

В данном разделе был проведен анализ предметной области, в результате чего было определено,
что визуальные конструкторы значительно упрощают процесс создания и запуска Telegram ботов,
однако, они имеет функциональные ограничения, которые препятствуют разработке продукта со сложной логикой работы.
Предметно-ориентированный язык позволяет расширить возможности визуального конструктора и создавать уникальных ботов,
в соответствии с пользовательскими потребностями.

Также было рассмотрено расширенное техническое задание,
в котором были определены цели и задачи разработки и основные требования к разрабатываемому программному продукту.

Основываясь на этом можно приступить к разработке предметно-ориентированного языка для визуального конструктора Telegram ботов,
так как рассмотренная проблема является актуальной.